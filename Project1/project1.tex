
\documentclass[letterpaper]{report}

%% ---------------------------------------------------
%% 634format specifies the format of our reports
%% ---------------------------------------------------
\usepackage{634format}

%% ---------------------------------------------------
%% enumerate 
%% ---------------------------------------------------
\usepackage{enumerate}

%% ---------------------------------------------------
%% listings is used for including our source code in reports
%% ---------------------------------------------------
\usepackage{listings}
\usepackage{textcomp}
\usepackage{indentfirst}

%% ---------------------------------------------------
%% Packages for math environments
%% ---------------------------------------------------
\usepackage{amsmath}

%% ---------------------------------------------------
%% Packages for URLs and hotlinks in the table of contents
%% and symbolic cross references using \ref
%% ---------------------------------------------------
\usepackage{hyperref}

%% ---------------------------------------------------
%% Packages for using HOL-generated macros and displays
%% ---------------------------------------------------
\usepackage{holtex}
\usepackage{holtexbasic}
\input{commands}

%% ---------------------------------------------------
%% Package for a table that will break across pages
%% ---------------------------------------------------
\usepackage{longtable}

%% ---------------------------------------------------
%% Packages to bring the contents of other files in
%% ---------------------------------------------------
\usepackage{import}
\usepackage{moreverb}

%% ---------------------------------------------------
%% parameter "language" is set to "ML"
%% ---------------------------------------------------
\lstset{language=ML}



\title{Project 1}
\author{\textbf{Alfred Murabito}}
\date{\today}

\begin{document}
\maketitle{}

\begin{abstract}
\noindent{}This report details results for the following exercise from \textit{Certified Security by Design Using Higher Order Logic}: 2.5.1, 3.4.1, and 3.4.2. We go over function definitions in  ML 
as well as ML types and type errors.
\end{abstract}

\newpage

\textbf{Acknowledgments}: I received no assistance with this project.

\newpage

\tableofcontents

\newpage

\chapter{Executive Summary}
\label{sec:executive-summary}

All requirements for this project have been satisfied.  A description and the results of each exercise are detailed in this project report. Each exercise is detailed with a problem statement, relevant
code, and execution transcripts as listed in requirements.

\newpage

\chapter{Exercise 2.5.1}
\label{cha:exercise-2.5.1}

\section{Problem Statement}
We define a function in ML and evaluate it in HOL. The function
returns a tuple of the sum and product of two numbers.

\section{Relevant Code}

\begin{lstlisting}
  fun timesPlus x y = (x*y, x+y);
\end{lstlisting}

\section{Test Cases}

\noindent Test cases below as listed in the project1 requirements.

\begin{lstlisting}
(* Test Cases *)
timesPlus 100 27;
timesPlus 10 26;
timesPlus 1 25;
timesPlus 2 24;
timesPlus 30 23;
timesPlus 50 200;
\end{lstlisting}

\noindent Reults below:
\verbatiminput{ML/ex251.trans}

\newpage

\chapter{Exercise 3.4.1}
\label{cha:exercise-3.4.1}

\section{Problem Statement}
We explore val definitons using pattern matching on tuples and lists in 
this example. The execution of our value declarations shown below.

\section{Relevant Code}

\begin{lstlisting}
(* Part A *)
val listA = [(0,"Alice"),(1,"Bob"),(3,"Carol"),(4,"Dan")]

(* Part B *)
val e1B::listB = listA;

(* Part C *)
val (e1C1,e1C2) = e1B;
val [e1C3,e1C4,e1C5] = listB;
\end{lstlisting}

\section{Test Cases}

Results below:
\verbatiminput{ML/ex341.trans}

\newpage

\chapter{Exercise 3.4.2}
\label{cha:exercise-3.4.2}

\section{Problem Statement}

In this exercise, we execute ML code and observe a couple type errors that occur in the following
value definitons.

\section{Relevant Code}

\begin{lstlisting}
val (x1,x2,x3) = (1,true,"Alice");
val pair1 = (x1,x3);
val list1 = [0,x1,2];
val list2 = [x2,x1];
val list3 = (1 :: [x3]);
\end{lstlisting}


\section{Test Cases}

The last two definitons produce errors because (``x2'' and ``x1'') and (1 and ``x3'') 
are different types and a list must have all elements be of the same type. 

\noindent Results below:
\verbatiminput{ML/ex342.trans}

\appendix

\chapter{Exercise 2.5.1 Source Code}
\label{cha:exerc-2.5.1-source}

\verbatiminput{ML/ex251.sml}

\newpage

\chapter{Exercise 3.4.1 Source Code}
\label{cha:exerc-3.4.1-source}

\verbatiminput{ML/ex341.sml}

\newpage

\chapter{Exercise 3.4.2 Source Code}
\label{cha:exerc-3.4.2-source}

\verbatiminput{ML/ex342.sml}

\end{document}
