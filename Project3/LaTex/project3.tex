
\documentclass[letterpaper]{report}

%% ---------------------------------------------------
%% 634format specifies the format of our reports
%% ---------------------------------------------------
\usepackage{634format}

%% ---------------------------------------------------
%% enumerate 
%% ---------------------------------------------------
\usepackage{enumerate}

%% ---------------------------------------------------
%% listings is used for including our source code in reports
%% ---------------------------------------------------
\usepackage{listings}
\usepackage{textcomp}
\usepackage{indentfirst}

%% ---------------------------------------------------
%% Packages for math environments
%% ---------------------------------------------------
\usepackage{amsmath}

%% ---------------------------------------------------
%% Packages for URLs and hotlinks in the table of contents
%% and symbolic cross references using \ref
%% ---------------------------------------------------
\usepackage{hyperref}

%% ---------------------------------------------------
%% Packages for using HOL-generated macros and displays
%% ---------------------------------------------------
\usepackage{holtex}
\usepackage{holtexbasic}
% =====================================================================
%
% Macros for typesetting the HOL system manual
%
% =====================================================================

% ---------------------------------------------------------------------
% Abbreviations for words and phrases
% ---------------------------------------------------------------------

\newcommand\TUTORIAL{{\footnotesize\sl TUTORIAL}}
\newcommand\DESCRIPTION{{\footnotesize\sl DESCRIPTION}}
\newcommand\REFERENCE{{\footnotesize\sl REFERENCE}}
\newcommand\LOGIC{{\footnotesize\sl LOGIC}}
\newcommand\LIBRARIES{{\footnotesize\sl LIBRARIES}}
\usepackage{textcomp}

\newcommand{\bs}{\texttt{\char'134}} % backslash
\newcommand{\lb}{\texttt{\char'173}} % left brace
\newcommand{\rb}{\texttt{\char'175}} % right brace
\newcommand{\td}{\texttt{\char'176}} % tilde
\newcommand{\lt}{\texttt{\char'74}} % less than
\newcommand{\gt}{\texttt{\char'76}} % greater than
\newcommand{\dol}{\texttt{\char'44}} % dollar
\newcommand{\pipe}{\texttt{\char'174}}
\newcommand{\apost}{\texttt{\textquotesingle}}
% double back quotes ``
\newcommand{\dq}{\texttt{\char'140\char'140}}
%These macros were included by slind:

\newcommand{\holquote}[1]{\dq#1\dq}

\def\HOL{\textsc{Hol}}
\def\holn{\HOL}  % i.e. hol n(inety-eight), no digits in
                 % macro names is a bit of a pain; deciding to do away
                 % with hol98 nomenclature means that we just want to
                 % write HOL for hol98.
\def\holnversion{Kananaskis-11}
\def\holnsversion{Kananaskis~11} % version with space rather than hyphen
\def\LCF{{\small LCF}}
\def\LCFLSM{{\small LCF{\kern-.2em}{\normalsize\_}{\kern0.1em}LSM}}
\def\PPL{{\small PP}{\kern-.095em}$\lambda$}
\def\PPLAMBDA{{\small PPLAMBDA}}
\def\ML{{\small ML}}
\def\holmake{\texttt{Holmake}}

\newcommand\ie{\mbox{\textit{i{.}e{.}}}}
\newcommand\eg{\mbox{\textit{e{.}g{.}}}}
\newcommand\viz{\mbox{viz{.}}}
\newcommand\adhoc{\mbox{\it ad hoc}}
\newcommand\etal{{\it et al.\/}}
% NOTE: \etc produces wrong spacing if used between sentences, that is
% like here \etc End such sentences with non-macro etc.
\newcommand\etc{\mbox{\textit{etc{.}}}}

% ---------------------------------------------------------------------
% Simple abbreviations and macros for mathematical typesetting
% ---------------------------------------------------------------------

\newcommand\fun{{\to}}
\newcommand\prd{{\times}}

\newcommand\conj{\ \wedge\ }
\newcommand\disj{\ \vee\ }
\newcommand\imp{ \Rightarrow }
\newcommand\eqv{\ \equiv\ }
\newcommand\cond{\rightarrow}
\newcommand\vbar{\mid}
\newcommand\turn{\ \vdash\ } % FIXME: "\ " results in extra space
\newcommand\hilbert{\varepsilon}
\newcommand\eqdef{\ \equiv\ }

\newcommand\natnums{\mbox{${\sf N}\!\!\!\!{\sf N}$}}
\newcommand\bools{\mbox{${\sf T}\!\!\!\!{\sf T}$}}

\newcommand\p{$\prime$}
\newcommand\f{$\forall$\ }
\newcommand\e{$\exists$\ }

\newcommand\orr{$\vee$\ }
\newcommand\negg{$\neg$\ }

\newcommand\arrr{$\rightarrow$}
\newcommand\hex{$\sharp $}

\newcommand{\uquant}[1]{\forall #1.\ }
\newcommand{\equant}[1]{\exists #1.\ }
\newcommand{\hquant}[1]{\hilbert #1.\ }
\newcommand{\iquant}[1]{\exists ! #1.\ }
\newcommand{\lquant}[1]{\lambda #1.\ }

\newcommand{\leave}[1]{\\[#1]\noindent}
\newcommand\entails{\mbox{\rule{.3mm}{4mm}\rule[2mm]{.2in}{.3mm}}}

% ---------------------------------------------------------------------
% Font-changing commands
% ---------------------------------------------------------------------

\newcommand{\theory}[1]{\hbox{{\small\tt #1}}}
\newcommand{\theoryimp}[1]{\texttt{#1}}

\newcommand{\con}[1]{{\sf #1}}
\newcommand{\rul}[1]{{\tt #1}}
\newcommand{\ty}[1]{\textsl{#1}}

\newcommand{\ml}[1]{\mbox{{\def\_{\char'137}\texttt{#1}}}}
\newcommand{\holtxt}[1]{\ml{#1}}
\newcommand\ms{\tt}
\newcommand{\s}[1]{{\small #1}}

\newcommand{\pin}[1]{{\bf #1}}
% FIXME: for multichar symbols \mathit should be used.
\def\m#1{\mbox{\normalsize$#1$}}

% ---------------------------------------------------------------------
% Abbreviations for particular mathematical constants etc.
% ---------------------------------------------------------------------

\newcommand\T{\con{T}}
\newcommand\F{\con{F}}
\newcommand\OneOne{\con{One\_One}}
\newcommand\OntoSubset{\con{Onto\_Subset}}
\newcommand\Onto{\con{Onto}}
\newcommand\TyDef{\con{Type\_Definition}}
\newcommand\Inv{\con{Inv}}
\newcommand\com{\con{o}}
\newcommand\Id{\con{I}}
\newcommand\MkPair{\con{Mk\_Pair}}
\newcommand\IsPair{\con{Is\_Pair}}
\newcommand\Fst{\con{Fst}}
\newcommand\Snd{\con{Snd}}
\newcommand\Suc{\con{Suc}}
\newcommand\Nil{\con{Nil}}
\newcommand\Cons{\con{Cons}}
\newcommand\Hd{\con{Hd}}
\newcommand\Tl{\con{Tl}}
\newcommand\Null{\con{Null}}
\newcommand\ListPrimRec{\con{List\_Prim\_Rec}}


\newcommand\SimpRec{\con{Simp\_Rec}}
\newcommand\SimpRecRel{\con{Simp\_Rec\_Rel}}
\newcommand\SimpRecFun{\con{Simp\_Rec\_Fun}}
\newcommand\PrimRec{\con{Prim\_Rec}}
\newcommand\PrimRecRel{\con{Prim\_Rec\_Rel}}
\newcommand\PrimRecFun{\con{Prim\_Rec\_Fun}}

\newcommand\bool{\ty{bool}}
\newcommand\num{\ty{num}}
\newcommand\ind{\ty{ind}}
\newcommand\lst{\ty{list}}

% ---------------------------------------------------------------------
% \minipagewidth = \textwidth minus 1.02 em
% ---------------------------------------------------------------------

\newlength{\minipagewidth}
\setlength{\minipagewidth}{\textwidth}
\addtolength{\minipagewidth}{-1.02em}

% ---------------------------------------------------------------------
% Environment for the items on the title page of a case study
% ---------------------------------------------------------------------

\newenvironment{inset}[1]{\noindent{\large\bf #1}\begin{list}%
{}{\setlength{\leftmargin}{\parindent}%
\setlength{\topsep}{-.1in}}\item }{\end{list}\vskip .4in}

% ---------------------------------------------------------------------
% Macros for little HOL sessions displayed in boxes.
%
% Usage: (1) \setcounter{sessioncount}{1} resets the session counter
%
%        (2) \begin{session}\begin{verbatim}
%             .
%              < lines from hol session >
%             .
%            \end{verbatim}\end{session}
%
%            typesets the session in a numbered box.
% ---------------------------------------------------------------------

\newlength{\hsbw}
\setlength{\hsbw}{\textwidth}
\addtolength{\hsbw}{-\arrayrulewidth}
\addtolength{\hsbw}{-\tabcolsep}
\newcommand\HOLSpacing{13pt}

\newcounter{sessioncount}
\setcounter{sessioncount}{0}

\newenvironment{session}{\begin{flushleft}
 \refstepcounter{sessioncount}
 \begin{tabular}{@{}|c@{}|@{}}\hline
 \begin{minipage}[b]{\hsbw}
 \vspace*{-.5pt}
 \begin{flushright}
 \rule{0.01in}{.15in}\rule{0.3in}{0.01in}\hspace{-0.35in}
 \raisebox{0.04in}{\makebox[0.3in][c]{\footnotesize\sl \thesessioncount}}
 \end{flushright}
 \vspace*{-.55in}
 \begingroup\small\baselineskip\HOLSpacing}{\endgroup\end{minipage}\\ \hline
 \end{tabular}
 \end{flushleft}}

% ---------------------------------------------------------------------
% Macro for boxed ML functions, etc.
%
% Usage: (1) \begin{holboxed}\begin{verbatim}
%               .
%               < lines giving names and types of mk functions >
%               .
%            \end{verbatim}\end{holboxed}
%
%            typesets the given lines in a box.
%
%            Conventions: lines are left-aligned under the "g" of begin,
%            and used to highlight primary reference for the ml function(s)
%            that appear in the box.
% ---------------------------------------------------------------------

\newenvironment{holboxed}{\begin{flushleft}
  \begin{tabular}{@{}|c@{}|@{}}\hline
  \begin{minipage}[b]{\hsbw}
% \vspace*{-.55in}
  \vspace*{.06in}
  \begingroup\small\baselineskip\HOLSpacing}{\endgroup\end{minipage}\\ \hline
  \end{tabular}
  \end{flushleft}}

% ---------------------------------------------------------------------
% Macro for unboxed ML functions, etc.
%
% Usage: (1) \begin{hol}\begin{verbatim}
%               .
%               < lines giving names and types of mk functions >
%               .
%            \end{verbatim}\end{hol}
%
%            typesets the given lines exactly like {boxed}, except there's
%            no box.
%
%            Conventions: lines are left-aligned under the "g" of begin,
%            and used to display ML code in verbatim, left aligned.
% ---------------------------------------------------------------------

\newenvironment{hol}{\begin{flushleft}
 \begin{tabular}{c@{}@{}}
 \begin{minipage}[b]{\hsbw}
% \vspace*{-.55in}
 \vspace*{.06in}
 \begingroup\small\baselineskip\HOLSpacing}{\endgroup\end{minipage}\\
 \end{tabular}
 \end{flushleft}}

% ---------------------------------------------------------------------
% Emphatic brackets
% ---------------------------------------------------------------------

\newcommand\leb{\lbrack\!\lbrack}
\newcommand\reb{\rbrack\!\rbrack}


% ---------------------------------------------------------------------
% Quotations
% ---------------------------------------------------------------------


%These macros were included by ap; they are used in Chapters 9 and 10
%of the HOL DESCRIPTION

\newcommand{\inds}%standard infinite set
 {\mbox{\rm I}}

\newcommand{\ch}%standard choice function
 {\mbox{\rm ch}}

\newcommand{\den}[1]%denotational brackets
 {[\![#1]\!]}

\newcommand{\two}%standard 2-element set
 {\mbox{\rm 2}}


%% ---------------------------------------------------
%% Package for a table that will break across pages
%% ---------------------------------------------------
\usepackage{longtable}

%% ---------------------------------------------------
%% Packages to bring the contents of other files in
%% ---------------------------------------------------
\usepackage{import}
\usepackage{moreverb}

%% ---------------------------------------------------
%% parameter "language" is set to "ML"
%% ---------------------------------------------------
\lstset{language=ML}

\newcommand{\HOLchapterEightDate}{22 March 2020}
\newcommand{\HOLchapterEightTime}{01:31}
\begin{SaveVerbatim}{HOLchapterEightTheoremsconjSymThm}
\HOLTokenTurnstile{} \HOLFreeVar{q} \HOLSymConst{\HOLTokenConj{}} \HOLFreeVar{p} \HOLSymConst{\HOLTokenEquiv{}} \HOLFreeVar{p} \HOLSymConst{\HOLTokenConj{}} \HOLFreeVar{q}
\end{SaveVerbatim}
\newcommand{\HOLchapterEightTheoremsconjSymThm}{\UseVerbatim{HOLchapterEightTheoremsconjSymThm}}
\begin{SaveVerbatim}{HOLchapterEightTheoremsconjSymThmAll}
\HOLTokenTurnstile{} \HOLSymConst{\HOLTokenForall{}}\HOLBoundVar{p} \HOLBoundVar{q}. \HOLBoundVar{q} \HOLSymConst{\HOLTokenConj{}} \HOLBoundVar{p} \HOLSymConst{\HOLTokenEquiv{}} \HOLBoundVar{p} \HOLSymConst{\HOLTokenConj{}} \HOLBoundVar{q}
\end{SaveVerbatim}
\newcommand{\HOLchapterEightTheoremsconjSymThmAll}{\UseVerbatim{HOLchapterEightTheoremsconjSymThmAll}}
\begin{SaveVerbatim}{HOLchapterEightTheoremsproblemOneThm}
\HOLTokenTurnstile{} \HOLFreeVar{p} \HOLSymConst{\HOLTokenImp{}} (\HOLFreeVar{p} \HOLSymConst{\HOLTokenImp{}} \HOLFreeVar{q}) \HOLSymConst{\HOLTokenImp{}} (\HOLFreeVar{q} \HOLSymConst{\HOLTokenImp{}} \HOLFreeVar{r}) \HOLSymConst{\HOLTokenImp{}} \HOLFreeVar{r}
\end{SaveVerbatim}
\newcommand{\HOLchapterEightTheoremsproblemOneThm}{\UseVerbatim{HOLchapterEightTheoremsproblemOneThm}}
\newcommand{\HOLchapterEightTheorems}{
\HOLThmTag{chapter8}{conjSymThm}\HOLchapterEightTheoremsconjSymThm
\HOLThmTag{chapter8}{conjSymThmAll}\HOLchapterEightTheoremsconjSymThmAll
\HOLThmTag{chapter8}{problem1Thm}\HOLchapterEightTheoremsproblemOneThm
}


\title{Project 3}
\author{\textbf{Alfred Murabito}}
\date{\today}

\begin{document}
\maketitle{}

\begin{abstract}
   \noindent{}This report details results for the following exercises from \textit{Certified Security by Design Using Higher Order Logic}: 7.3.1, 7.3.2, 7.3.3, 8.4.1, 8.4.2, and 8.4.3.  The purpose of this project is to gain skill in generating reproducible proofs and documentation using EmitTeX.  In addition, HOL Theories will be used to complete the exercises listed above.
\end{abstract}

\newpage

\textbf{Acknowledgments}: I received no assistance with this project.

\newpage

\tableofcontents


\newpage

\chapter{Executive Summary}
\label{sec:executive-summary}

All requirements for this project have been satisfied.  A description of each exercise from Chapter 7, the test cases and the results of each exercise as based on HOL's built-in functions are detailed in this project report.  

In addition, a HOL subfolder has been created and contains the proofs for the exercises in Chapter 8.  

\newpage

\chapter{Exercise 7.3.1}
\label{sec:ex-7-3-1}

\section{Problem Statement}
In this exercise we define a function that operates on terms of the form \textit{p /\ q ==> r} and returns \textit{p ==> q ==> r}.

\section{Relevant Code}
\label{sec:relevantCode}

\begin{lstlisting}
(* Solution to Exercise 7_3_1  *)

fun andImp2Imp term_i =
  let
    val (p_and_q, r) = dest_imp term_i
    val (p, q)  = dest_conj p_and_q
  in
    mk_imp (p, mk_imp (q, r))
end;
\end{lstlisting}


\section{Test Cases and Results}
\label{sec:tests}
\setcounter{sessioncount}{0}

\vspace{12px}
\indent Test cases for Exercise 7.3.1:
\begin{lstlisting}
andImp2Imp ``(p /\ q) ==> r``;
\end{lstlisting}


Results for Exercise 7.3.1:

\verbatiminput{../ML/ex731.trans}



\newpage


\chapter{Exercise 7.3.2}
\label{sec:ex-7-3-2}

\section{Problem Statement}
In this exercise we define a function that accepts terms of the form p ==> q ==> r and returns p /\ q ==> r. This function
is shown to reverse the effect of the function in 7.3.1.    


\section{Relevant Code}

\begin{lstlisting}
(* Exercise 7_3_2 *)
(* Turns terms of p ==> q ==> r to (p /\ q) ==> r *)

fun impImpAnd term_i =
let
  val (p, q_imp_r) = dest_imp term_i
  val (q, r) = dest_imp q_imp_r
in
  mk_imp (mk_conj (p, q), r)
end;

\end{lstlisting}

\section{Test Cases and Results}

We include test cases to show the function's behavior and the reversal of the function from 7.3.1.

\begin{lstlisting}
impImpAnd ``p ==> q ==> r``;
impImpAnd(andImp2Imp ``(p /\ q) ==> r``);
andImp2Imp(impImpAnd ``p ==> q ==> r``);
\end{lstlisting}

Results for Exercise 7.3.2:

\verbatiminput{../ML/ex732.trans}



\newpage


\chapter{Exercise 7.3.3}
\label{sec:ex-7-3-3}

\section{Problem Statement}
In this exercise we define a function that accepts terms of the form ~@x.P(x) and returns !x.~P(x).  

\section{Relevant Code}

\begin{lstlisting}
(Exercise 7_3_3  *)
(* Given ~?x.P(x) return !x.~P(x)

fun notExists term_i =
let
  val (x, F ) = dest_exists (dest_neg term_i)
in
  mk_forall (x, F )
end;
\end{lstlisting}

\section{Test Cases and Results}

Test cases for Exercise 7.3.3:
\begin{lstlisting}
notExists ``~?z.Q z``;
\end{lstlisting}

Results for Exercise 7.3.3:

\verbatiminput{../ML/ex733.trans}


\newpage

\chapter{Exercise 8.4.1}
\label{sec:ex-8-4-1}

\section{Problem Statement}
In this exercise we prove the theorem
 
\HOLchapterEightTheoremsproblemOneThm

\section{Relevant Code}

\begin{lstlisting}
(* Excercise 8_4_1 *)
val problem1Thm =
let
  val p_term = ``p:bool``
  val q_term = ``q:bool``
  val r_term = ``r:bool``
  val p_imp_q = mk_imp (p_term, q_term)
  val q_imp_r = mk_imp (q_term, r_term)
  val thm1 = ASSUME p_term
  val thm2 = ASSUME p_imp_q
  val thm3 = MP (ASSUME p_imp_q) (ASSUME p_term)
  val thm4 = MP (ASSUME q_imp_r) thm3
in
  DISCH p_term (DISCH p_imp_q (DISCH q_imp_r thm4))
end;
\end{lstlisting}

\section{Execution Transcript}

Transcript for Exercise 8.4.1:

\verbatiminput{../HOL/ex841.trans}



\newpage



\chapter{Exercise 8.4.2}
\label{sec:ex-8-4-2}

\section{Problem Statement}
In this exercise we prove the theorem

\HOLchapterEightTheoremsconjSymThm

\section{Relevant Code}

\begin{lstlisting}
(* Prove p /\ q <=> q /\ p *)
val conjSymThm =
let
  val p_term = ``p:bool``
  val q_term = ``q:bool``
  val p_and_q = mk_conj (p_term, q_term)
  val q_and_p = mk_conj (q_term, p_term)

  (* Prove -| p /\ q ==> q /\ p *)
  val thm1 = ASSUME p_and_q
  val thm2 = CONJUNCT1 thm1
  val thm3 = CONJUNCT2 thm1
  val thm4 = CONJ thm3 thm2
  val thm5 = DISCH p_and_q thm4

  (* Prove -| q /\ p ==> p /\ q *)
  val thm6 = ASSUME q_and_p
  val thm7 = CONJUNCT1 thm6
  val thm8 = CONJUNCT2 thm6
  val thm9 = CONJ thm8 thm7
  val thm10 = DISCH q_and_p thm9


in
  IMP_ANTISYM_RULE thm10 thm5
end;
\end{lstlisting}

\section{Execution Transcript}

Results for Exercise 8.4.2:

\verbatiminput{../HOL/ex842.trans}


\newpage


\chapter{Exercise 8.4.3}
\label{sec:ex-8-4-3}

\section{Problem Statement}

In this exercise we extend upon the previous theorem through generalization to 
prove the following theorem:

\HOLchapterEightTheoremsconjSymThmAll.

\section{Relevant Code}

\begin{lstlisting}
(* Prove p /\ q <=> q /\ p *)
val conjSymThmAll =
let
  val p_term = ``p:bool``
  val q_term = ``q:bool``
  val p_and_q = mk_conj (p_term, q_term)
  val q_and_p = mk_conj (q_term, p_term)

  (* Prove -| p /\ q ==> q /\ p *)
  val thm1 = ASSUME p_and_q
  val thm2 = CONJUNCT1 thm1
  val thm3 = CONJUNCT2 thm1
  val thm4 = CONJ thm3 thm2
  val thm5 = DISCH p_and_q thm4

  (* Prove -| q /\ p ==> p /\ q *)
  val thm6 = ASSUME q_and_p
  val thm7 = CONJUNCT1 thm6
  val thm8 = CONJUNCT2 thm6
  val thm9 = CONJ thm8 thm7
  val thm10 = DISCH q_and_p thm9
  val thm11 = IMP_ANTISYM_RULE thm10 thm5

in
  GENL [p_term, q_term] thm11
end;
\end{lstlisting}

\section{Execution Transcript}

Results for Exercise 8.4.3:

\verbatiminput{../HOL/ex-843.trans}


\newpage


\appendix

\chapter{Chapter7 Answers Source Code}
\label{cha:chapt-answ-source}

\verbatiminput{../ML/chapter7Answers.sml}

\newpage

\chapter{Chapter8Script Source Code}
\label{cha:chapt-source-code}


\verbatiminput{../HOL/chapter8Script.sml}


\end{document}
