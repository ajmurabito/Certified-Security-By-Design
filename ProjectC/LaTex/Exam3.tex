\documentclass[11pt, twoside]{article}

%% ---------------------------------------------------
%% 634format specifies the format of our reports
%% ---------------------------------------------------
\usepackage{634format}

%% ---------------------------------------------------
%% enumerate 
%% ---------------------------------------------------
\usepackage{enumerate}

%% ---------------------------------------------------
%% listings is used for including our source code in reports
%% ---------------------------------------------------
\usepackage{listings}
\usepackage{textcomp}
\usepackage{indentfirst}

%% ---------------------------------------------------
%% Packages for math environments
%% ---------------------------------------------------
\usepackage{amsmath}

%% ---------------------------------------------------
%% Packages for URLs and hotlinks in the table of contents
%% and symbolic cross references using \ref
%% ---------------------------------------------------
\usepackage{hyperref}

%% ---------------------------------------------------
%% Packages for using HOL-generated macros and displays
%% ---------------------------------------------------
\usepackage{holtex}
\usepackage{holtexbasic}
% =====================================================================
%
% Macros for typesetting the HOL system manual
%
% =====================================================================

% ---------------------------------------------------------------------
% Abbreviations for words and phrases
% ---------------------------------------------------------------------

\newcommand\TUTORIAL{{\footnotesize\sl TUTORIAL}}
\newcommand\DESCRIPTION{{\footnotesize\sl DESCRIPTION}}
\newcommand\REFERENCE{{\footnotesize\sl REFERENCE}}
\newcommand\LOGIC{{\footnotesize\sl LOGIC}}
\newcommand\LIBRARIES{{\footnotesize\sl LIBRARIES}}
\usepackage{textcomp}

\newcommand{\bs}{\texttt{\char'134}} % backslash
\newcommand{\lb}{\texttt{\char'173}} % left brace
\newcommand{\rb}{\texttt{\char'175}} % right brace
\newcommand{\td}{\texttt{\char'176}} % tilde
\newcommand{\lt}{\texttt{\char'74}} % less than
\newcommand{\gt}{\texttt{\char'76}} % greater than
\newcommand{\dol}{\texttt{\char'44}} % dollar
\newcommand{\pipe}{\texttt{\char'174}}
\newcommand{\apost}{\texttt{\textquotesingle}}
% double back quotes ``
\newcommand{\dq}{\texttt{\char'140\char'140}}
%These macros were included by slind:

\newcommand{\holquote}[1]{\dq#1\dq}

\def\HOL{\textsc{Hol}}
\def\holn{\HOL}  % i.e. hol n(inety-eight), no digits in
                 % macro names is a bit of a pain; deciding to do away
                 % with hol98 nomenclature means that we just want to
                 % write HOL for hol98.
\def\holnversion{Kananaskis-11}
\def\holnsversion{Kananaskis~11} % version with space rather than hyphen
\def\LCF{{\small LCF}}
\def\LCFLSM{{\small LCF{\kern-.2em}{\normalsize\_}{\kern0.1em}LSM}}
\def\PPL{{\small PP}{\kern-.095em}$\lambda$}
\def\PPLAMBDA{{\small PPLAMBDA}}
\def\ML{{\small ML}}
\def\holmake{\texttt{Holmake}}

\newcommand\ie{\mbox{\textit{i{.}e{.}}}}
\newcommand\eg{\mbox{\textit{e{.}g{.}}}}
\newcommand\viz{\mbox{viz{.}}}
\newcommand\adhoc{\mbox{\it ad hoc}}
\newcommand\etal{{\it et al.\/}}
% NOTE: \etc produces wrong spacing if used between sentences, that is
% like here \etc End such sentences with non-macro etc.
\newcommand\etc{\mbox{\textit{etc{.}}}}

% ---------------------------------------------------------------------
% Simple abbreviations and macros for mathematical typesetting
% ---------------------------------------------------------------------

\newcommand\fun{{\to}}
\newcommand\prd{{\times}}

\newcommand\conj{\ \wedge\ }
\newcommand\disj{\ \vee\ }
\newcommand\imp{ \Rightarrow }
\newcommand\eqv{\ \equiv\ }
\newcommand\cond{\rightarrow}
\newcommand\vbar{\mid}
\newcommand\turn{\ \vdash\ } % FIXME: "\ " results in extra space
\newcommand\hilbert{\varepsilon}
\newcommand\eqdef{\ \equiv\ }

\newcommand\natnums{\mbox{${\sf N}\!\!\!\!{\sf N}$}}
\newcommand\bools{\mbox{${\sf T}\!\!\!\!{\sf T}$}}

\newcommand\p{$\prime$}
\newcommand\f{$\forall$\ }
\newcommand\e{$\exists$\ }

\newcommand\orr{$\vee$\ }
\newcommand\negg{$\neg$\ }

\newcommand\arrr{$\rightarrow$}
\newcommand\hex{$\sharp $}

\newcommand{\uquant}[1]{\forall #1.\ }
\newcommand{\equant}[1]{\exists #1.\ }
\newcommand{\hquant}[1]{\hilbert #1.\ }
\newcommand{\iquant}[1]{\exists ! #1.\ }
\newcommand{\lquant}[1]{\lambda #1.\ }

\newcommand{\leave}[1]{\\[#1]\noindent}
\newcommand\entails{\mbox{\rule{.3mm}{4mm}\rule[2mm]{.2in}{.3mm}}}

% ---------------------------------------------------------------------
% Font-changing commands
% ---------------------------------------------------------------------

\newcommand{\theory}[1]{\hbox{{\small\tt #1}}}
\newcommand{\theoryimp}[1]{\texttt{#1}}

\newcommand{\con}[1]{{\sf #1}}
\newcommand{\rul}[1]{{\tt #1}}
\newcommand{\ty}[1]{\textsl{#1}}

\newcommand{\ml}[1]{\mbox{{\def\_{\char'137}\texttt{#1}}}}
\newcommand{\holtxt}[1]{\ml{#1}}
\newcommand\ms{\tt}
\newcommand{\s}[1]{{\small #1}}

\newcommand{\pin}[1]{{\bf #1}}
% FIXME: for multichar symbols \mathit should be used.
\def\m#1{\mbox{\normalsize$#1$}}

% ---------------------------------------------------------------------
% Abbreviations for particular mathematical constants etc.
% ---------------------------------------------------------------------

\newcommand\T{\con{T}}
\newcommand\F{\con{F}}
\newcommand\OneOne{\con{One\_One}}
\newcommand\OntoSubset{\con{Onto\_Subset}}
\newcommand\Onto{\con{Onto}}
\newcommand\TyDef{\con{Type\_Definition}}
\newcommand\Inv{\con{Inv}}
\newcommand\com{\con{o}}
\newcommand\Id{\con{I}}
\newcommand\MkPair{\con{Mk\_Pair}}
\newcommand\IsPair{\con{Is\_Pair}}
\newcommand\Fst{\con{Fst}}
\newcommand\Snd{\con{Snd}}
\newcommand\Suc{\con{Suc}}
\newcommand\Nil{\con{Nil}}
\newcommand\Cons{\con{Cons}}
\newcommand\Hd{\con{Hd}}
\newcommand\Tl{\con{Tl}}
\newcommand\Null{\con{Null}}
\newcommand\ListPrimRec{\con{List\_Prim\_Rec}}


\newcommand\SimpRec{\con{Simp\_Rec}}
\newcommand\SimpRecRel{\con{Simp\_Rec\_Rel}}
\newcommand\SimpRecFun{\con{Simp\_Rec\_Fun}}
\newcommand\PrimRec{\con{Prim\_Rec}}
\newcommand\PrimRecRel{\con{Prim\_Rec\_Rel}}
\newcommand\PrimRecFun{\con{Prim\_Rec\_Fun}}

\newcommand\bool{\ty{bool}}
\newcommand\num{\ty{num}}
\newcommand\ind{\ty{ind}}
\newcommand\lst{\ty{list}}

% ---------------------------------------------------------------------
% \minipagewidth = \textwidth minus 1.02 em
% ---------------------------------------------------------------------

\newlength{\minipagewidth}
\setlength{\minipagewidth}{\textwidth}
\addtolength{\minipagewidth}{-1.02em}

% ---------------------------------------------------------------------
% Environment for the items on the title page of a case study
% ---------------------------------------------------------------------

\newenvironment{inset}[1]{\noindent{\large\bf #1}\begin{list}%
{}{\setlength{\leftmargin}{\parindent}%
\setlength{\topsep}{-.1in}}\item }{\end{list}\vskip .4in}

% ---------------------------------------------------------------------
% Macros for little HOL sessions displayed in boxes.
%
% Usage: (1) \setcounter{sessioncount}{1} resets the session counter
%
%        (2) \begin{session}\begin{verbatim}
%             .
%              < lines from hol session >
%             .
%            \end{verbatim}\end{session}
%
%            typesets the session in a numbered box.
% ---------------------------------------------------------------------

\newlength{\hsbw}
\setlength{\hsbw}{\textwidth}
\addtolength{\hsbw}{-\arrayrulewidth}
\addtolength{\hsbw}{-\tabcolsep}
\newcommand\HOLSpacing{13pt}

\newcounter{sessioncount}
\setcounter{sessioncount}{0}

\newenvironment{session}{\begin{flushleft}
 \refstepcounter{sessioncount}
 \begin{tabular}{@{}|c@{}|@{}}\hline
 \begin{minipage}[b]{\hsbw}
 \vspace*{-.5pt}
 \begin{flushright}
 \rule{0.01in}{.15in}\rule{0.3in}{0.01in}\hspace{-0.35in}
 \raisebox{0.04in}{\makebox[0.3in][c]{\footnotesize\sl \thesessioncount}}
 \end{flushright}
 \vspace*{-.55in}
 \begingroup\small\baselineskip\HOLSpacing}{\endgroup\end{minipage}\\ \hline
 \end{tabular}
 \end{flushleft}}

% ---------------------------------------------------------------------
% Macro for boxed ML functions, etc.
%
% Usage: (1) \begin{holboxed}\begin{verbatim}
%               .
%               < lines giving names and types of mk functions >
%               .
%            \end{verbatim}\end{holboxed}
%
%            typesets the given lines in a box.
%
%            Conventions: lines are left-aligned under the "g" of begin,
%            and used to highlight primary reference for the ml function(s)
%            that appear in the box.
% ---------------------------------------------------------------------

\newenvironment{holboxed}{\begin{flushleft}
  \begin{tabular}{@{}|c@{}|@{}}\hline
  \begin{minipage}[b]{\hsbw}
% \vspace*{-.55in}
  \vspace*{.06in}
  \begingroup\small\baselineskip\HOLSpacing}{\endgroup\end{minipage}\\ \hline
  \end{tabular}
  \end{flushleft}}

% ---------------------------------------------------------------------
% Macro for unboxed ML functions, etc.
%
% Usage: (1) \begin{hol}\begin{verbatim}
%               .
%               < lines giving names and types of mk functions >
%               .
%            \end{verbatim}\end{hol}
%
%            typesets the given lines exactly like {boxed}, except there's
%            no box.
%
%            Conventions: lines are left-aligned under the "g" of begin,
%            and used to display ML code in verbatim, left aligned.
% ---------------------------------------------------------------------

\newenvironment{hol}{\begin{flushleft}
 \begin{tabular}{c@{}@{}}
 \begin{minipage}[b]{\hsbw}
% \vspace*{-.55in}
 \vspace*{.06in}
 \begingroup\small\baselineskip\HOLSpacing}{\endgroup\end{minipage}\\
 \end{tabular}
 \end{flushleft}}

% ---------------------------------------------------------------------
% Emphatic brackets
% ---------------------------------------------------------------------

\newcommand\leb{\lbrack\!\lbrack}
\newcommand\reb{\rbrack\!\rbrack}


% ---------------------------------------------------------------------
% Quotations
% ---------------------------------------------------------------------


%These macros were included by ap; they are used in Chapters 9 and 10
%of the HOL DESCRIPTION

\newcommand{\inds}%standard infinite set
 {\mbox{\rm I}}

\newcommand{\ch}%standard choice function
 {\mbox{\rm ch}}

\newcommand{\den}[1]%denotational brackets
 {[\![#1]\!]}

\newcommand{\two}%standard 2-element set
 {\mbox{\rm 2}}


%% ---------------------------------------------------
%% Package for a table that will break across pages
%% ---------------------------------------------------
\usepackage{longtable}

%% ---------------------------------------------------
%% Packages to bring the contents of other files in
%% ---------------------------------------------------
\usepackage{import}
\usepackage{moreverb}

%% ---------------------------------------------------
%% parameter "language" is set to "ML"
%% ---------------------------------------------------
\lstset{language=ML}

\makeindex

\newcommand{\HOLexamThreeDate}{19 March 2020}
\newcommand{\HOLexamThreeTime}{06:15}
\begin{SaveVerbatim}{HOLexamThreeDatatypescommands}
\HOLFreeVar{commands} = \HOLConst{go} \HOLTokenBar{} \HOLConst{launch}
\end{SaveVerbatim}
\newcommand{\HOLexamThreeDatatypescommands}{\UseVerbatim{HOLexamThreeDatatypescommands}}
\begin{SaveVerbatim}{HOLexamThreeDatatypeskeyPrinc}
\HOLFreeVar{keyPrinc} = \HOLConst{Staff} \HOLTyOp{people} \HOLTokenBar{} \HOLConst{Role} \HOLTyOp{roles} \HOLTokenBar{} \HOLConst{Ap} \HOLTyOp{num}
\end{SaveVerbatim}
\newcommand{\HOLexamThreeDatatypeskeyPrinc}{\UseVerbatim{HOLexamThreeDatatypeskeyPrinc}}
\begin{SaveVerbatim}{HOLexamThreeDatatypespeople}
\HOLFreeVar{people} = \HOLConst{Alice} \HOLTokenBar{} \HOLConst{Bob}
\end{SaveVerbatim}
\newcommand{\HOLexamThreeDatatypespeople}{\UseVerbatim{HOLexamThreeDatatypespeople}}
\begin{SaveVerbatim}{HOLexamThreeDatatypesprincipals}
\HOLFreeVar{principals} = \HOLConst{PR} \HOLTyOp{keyPrinc} \HOLTokenBar{} \HOLConst{Key} \HOLTyOp{keyPrinc}
\end{SaveVerbatim}
\newcommand{\HOLexamThreeDatatypesprincipals}{\UseVerbatim{HOLexamThreeDatatypesprincipals}}
\begin{SaveVerbatim}{HOLexamThreeDatatypesroles}
\HOLFreeVar{roles} = \HOLConst{Employee} \HOLTokenBar{} \HOLConst{Relay}
\end{SaveVerbatim}
\newcommand{\HOLexamThreeDatatypesroles}{\UseVerbatim{HOLexamThreeDatatypesroles}}
\newcommand{\HOLexamThreeDatatypes}{
\HOLexamThreeDatatypescommands\HOLexamThreeDatatypeskeyPrinc\HOLexamThreeDatatypespeople\HOLexamThreeDatatypesprincipals\HOLexamThreeDatatypesroles}
\begin{SaveVerbatim}{HOLexamThreeTheoremsRelayRuleLaunch}
\HOLTokenTurnstile{} (\HOLFreeVar{M}\HOLSymConst{,}\HOLFreeVar{Oi}\HOLSymConst{,}\HOLFreeVar{Os}) \HOLConst{sat} \HOLConst{Name} (\HOLConst{PR} (\HOLConst{Role} \HOLConst{Employee})) \HOLConst{controls} \HOLConst{prop} \HOLConst{go} \HOLSymConst{\HOLTokenImp{}}
   (\HOLFreeVar{M}\HOLSymConst{,}\HOLFreeVar{Oi}\HOLSymConst{,}\HOLFreeVar{Os}) \HOLConst{sat}
   \HOLConst{reps} (\HOLConst{Name} (\HOLConst{PR} (\HOLConst{Staff} \HOLConst{Alice}))) (\HOLConst{Name} (\HOLConst{PR} (\HOLConst{Role} \HOLConst{Employee})))
     (\HOLConst{prop} \HOLConst{go}) \HOLSymConst{\HOLTokenImp{}}
   (\HOLFreeVar{M}\HOLSymConst{,}\HOLFreeVar{Oi}\HOLSymConst{,}\HOLFreeVar{Os}) \HOLConst{sat}
   \HOLConst{Name} (\HOLConst{Key} (\HOLConst{Staff} \HOLConst{Alice})) \HOLConst{quoting}
   \HOLConst{Name} (\HOLConst{PR} (\HOLConst{Role} \HOLConst{Employee})) \HOLConst{says} \HOLConst{prop} \HOLConst{go} \HOLSymConst{\HOLTokenImp{}}
   (\HOLFreeVar{M}\HOLSymConst{,}\HOLFreeVar{Oi}\HOLSymConst{,}\HOLFreeVar{Os}) \HOLConst{sat} \HOLConst{prop} \HOLConst{go} \HOLConst{impf} \HOLConst{prop} \HOLConst{launch} \HOLSymConst{\HOLTokenImp{}}
   (\HOLFreeVar{M}\HOLSymConst{,}\HOLFreeVar{Oi}\HOLSymConst{,}\HOLFreeVar{Os}) \HOLConst{sat}
   \HOLConst{Name} (\HOLConst{Key} (\HOLConst{Staff} \HOLConst{Alice})) \HOLConst{speaks_for} \HOLConst{Name} (\HOLConst{PR} (\HOLConst{Staff} \HOLConst{Alice})) \HOLSymConst{\HOLTokenImp{}}
   (\HOLFreeVar{M}\HOLSymConst{,}\HOLFreeVar{Oi}\HOLSymConst{,}\HOLFreeVar{Os}) \HOLConst{sat}
   \HOLConst{Name} (\HOLConst{Key} (\HOLConst{Staff} \HOLConst{Bob})) \HOLConst{quoting} \HOLConst{Name} (\HOLConst{PR} (\HOLConst{Role} \HOLConst{Relay})) \HOLConst{says}
   \HOLConst{prop} \HOLConst{launch}
\end{SaveVerbatim}
\newcommand{\HOLexamThreeTheoremsRelayRuleLaunch}{\UseVerbatim{HOLexamThreeTheoremsRelayRuleLaunch}}
\newcommand{\HOLexamThreeTheorems}{
\HOLThmTag{exam3}{RelayRuleLaunch}\HOLexamThreeTheoremsRelayRuleLaunch
}


\title{Exam 3 Report}
\author{Alfred Murabito}
\date{\today}

\begin{document}

\maketitle{}

\newpage

\begin{abstract}
This report answers the three questions in the CIS 634 Exam 3. The 
first question is answered in the code in the HOL/exam1Script.sml
source file. The steps involving the shared-key encyrption scheme where a 
new key is generated for every message is answered in question 2. Question
3 involves what needs to be done to facilitate using HOL in artiifical
intelligence. 
\end{abstract}

\newpage

\textbf{Acknowledgments}: I received no assistance with this exercise.

\newpage

\tableofcontents

\newpage

\section{Executive Summary}
\label{sec:executive-summary}



All requirements for this assignment have been satisfied. A formulated proof is given for the derived inference rule
for the accescc control scheme described in problem 1. In addition, responses are given for the encryption scheme for question 2
and the question on theroem provers and aritificial intelligence of question 3.

\newpage

\section{Question 1}
\label{sec:question-1}

\subsection{Problem Statement}
\label{sec:problem-statement}

In this access control scheme, Alice and Bob take on the roles of Employee and Relay respectively. 
An inference rule is defined for the Relay to follow whenever it receives a message. If the conditions
in the inference rule is met, Bob will encrypt a launch message with his key and send it out 
as the Relay. I assume a certificate authority is not needed in this access control scenario since
Alice and Bob will be part of the same organization so excanging public keys legitimately should not 
be as much of an isssue.

The Inference rule is below.



\lstset{frameround=fftt}
\begin{lstlisting}

   Role Employee controls prop go ==>
   Alice Reps Employee on go ==>
   Key Alice speaks_for Alice ==>
   Key Alice says prop go ==>
   prop go impf prop launch ==>
   Key Bob quoting Role Relay says prop launch

\end{lstlisting}



HOL datatypes and theorems for problem 1 below.

\HOLexamThreeDatatypes

\HOLexamThreeTheorems

\subsection{Relevant Code}
\label{sec:relevant-code}

The proof of the derived inference rule involves using the \textit{PAT_ASSUM} to inject hyptothesises into
the proof until the goal is derived. The main theories used in the tactics are produced using the assumptions
and the \textit{CONTROLS}, \textit{QUOTING}, \textit{REPS}, and \textit{SPEAKS_FOR} inference rules.

\lstset{frameround=fftt}
\begin{lstlisting}

val RelayRuleLaunch =
TAC_PROOF(
([],
``(M,Oi,Os) sat Name (PR (Role Employee)) controls prop go ==>
  (M,Oi,Os) sat
  reps (Name (PR (Staff Alice))) (Name (PR (Role Employee)))
  (prop go) ==>
  (M,Oi,Os) sat
  Name (Key (Staff Alice)) quoting
  Name (PR (Role Employee)) says prop go ==>
  (M,Oi,Os) sat prop go impf prop launch ==>
  (M,Oi,Os) sat
  Name (Key (Staff Alice)) speaks_for Name (PR (Staff Alice)) ==>
  (M,Oi,Os) sat
  Name (Key (Staff Bob)) quoting Name (PR (Role Relay)) says
  prop launch``),
REPEAT STRIP_TAC THEN
ACL_SAYS_TAC THEN

PAT_ASSUM ``(M,Oi,Os) sat Name (Key (Staff Alice)) quoting
            Name (PR (Role Employee)) says prop go``
  (fn th => ASSUME_TAC (QUOTING_LR th)) THEN
		  
PAT_ASSUM ``(M,Oi,Os) sat
            Name (Key (Staff Alice)) speaks_for Name (PR (Staff Alice))``
  (fn th1 =>
    (PAT_ASSUM
    ``(M,Oi,Os) sat Name (Key (Staff Alice)) says
      Name (PR (Role Employee)) says (prop go)``
    (fn th2 => ASSUME_TAC(SPEAKS_FOR th1 th2)))) THEN

PAT_ASSUM ``(M,Oi,Os) sat
            Name (PR (Staff Alice)) says Name (PR (Role Employee)) says
            (prop go)``
  (fn th => ASSUME_TAC (QUOTING_RL th)) THEN

PAT_ASSUM ``(M,Oi,Os) sat
           reps (Name (PR (Staff Alice))) (Name (PR (Role Employee)))(prop go)``
  (fn th1 =>
    (PAT_ASSUM ``(M,Oi,Os) sat
    Name (PR (Staff Alice)) quoting Name (PR (Role Employee)) says
    (prop go)``
      (fn th2 =>
        (PAT_ASSUM ``(M,Oi,Os) sat Name (PR (Role Employee)) controls (prop go)``
	  (fn th3 => ASSUME_TAC (REPS th1 th2 th3)))))) THEN

PAT_ASSUM ``(M,Oi,Os) sat (prop go)`` (fn th1 =>
  (PAT_ASSUM ``(M,Oi,Os) sat prop go impf prop launch``
    (fn th2 => ASSUME_TAC (ACL_MP th1 th2)))) THEN

ASM_REWRITE_TAC[]);

\end{lstlisting}

\subsection{Execution Transcript}
\label{sec:execution-transcript}

\verbatiminput{../HOL/exam3.trans}

\newpage

\section{Question 2}
\label{sec:question-2}

Brian want to communicate securely using a different symmetric key for every message sent. In
addition, they must follow corporate policy and have all messages be transferred using a secure relay.
To accomplish this, they first get a set of n keys from the TA authority which also gives the keys to the secure 
relay. 

The set of keys for Alice is KA1,KA2,....KAn with published certificate encrypt(Kta, [KA1,KA2...KAN]Alice). 
Not explicitely stated above but key identifiers for ALice and the relay will be also be sent and included in the
certificate. Similar keys and certificate will be produced and given to Bob. In this scheme, Bob and Alice will exchange their
key identifiers with each other.

To initiate the secure channel, Alice will send her key hint pair(for key Ka), Bob's published certificate, 
as well as a key indentifier for Bob(Kb) to the relay. Subsequent messages to the Relay will feature
a different set of key identifiers that were received from the trusted authority. Multiple keys from the TA were needed since
if only one was used by both Alice and Bob to communicate with the relay, then if that key was exposed an attacker could uncover
every shared key generated by the relay for Alice and Bob to use.

The relay will use the TA's key in order to verify that the certificate and the sent identifier belongs to Bob. Once it verifies the 
ceritificate and key identifier for Kb is Bob's, it will generate a new key and send the identifier \textit{KB = encrypt(Kb,K)} as well 
as identifier for Kb to Bob. It will also send \textit{KA = encrypt(Ka,K)} to Alice with the identifier for Ka. Alice and Bob can now
get the shared key K by decrypting with their respective keys and Alice can send an encrypted message to Bob that he can decrypt with K. 

To authenticate that the message came from Alice, Alice will neeed to have sent her TA certiificate with the relay with the rest of
the initial message sent to the relay. The relay will decrypt using Kta then it will know that the key identifier Ka belongs to
Alice. It can then generate the certificate \textit{encrypt(Kb,[KB,KA,Alice])} paired with Kb's indentifier. When Bob gets this certificate,
he can verify with his key that the key identifier KB (as well as the new key K) was initiated from Alice since he trusts the relay and
the TA. He therefore knows messages that he can decrypt with K originated from Alice.


\newpage

\section{Question 3}
\label{sec:question-3}

To facilitate the use of theorem provers like HOL in artiificial intelligence, the intelligence will need to be build on verly large libraries
made initially by people. Due to the difficulty in desigining automated proofs for higher order logic sytems, very extensive libraries possibly made with the
assistance of ATPs will be needed. In addition, there would need to be a common ground and methodology established between all of the theorem provers. Without a 
general model provided, it will be hard for artiificial intelligence algorithms to work will all the different theorem provers.


\HOLindex

\appendix{}

\section{Source For exam3Script.sml}
\label{sec:source-exam3scr}


The following code is from \emph{HOL/exam3Script.sml}
\lstinputlisting{../HOL/exam3Script.sml}


\end{document}
