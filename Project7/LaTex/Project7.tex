\documentclass[11pt, twoside]{article}

%% ---------------------------------------------------
%% 634format specifies the format of our reports
%% ---------------------------------------------------
\usepackage{634format}

%% ---------------------------------------------------
%% enumerate 
%% ---------------------------------------------------
\usepackage{enumerate}

%% ---------------------------------------------------
%% listings is used for including our source code in reports
%% ---------------------------------------------------
\usepackage{listings}
\usepackage{textcomp}
\usepackage{indentfirst}

%% ---------------------------------------------------
%% Packages for math environments
%% ---------------------------------------------------
\usepackage{amsmath}

%% ---------------------------------------------------
%% Packages for URLs and hotlinks in the table of contents
%% and symbolic cross references using \ref
%% ---------------------------------------------------
\usepackage{hyperref}

%% ---------------------------------------------------
%% Packages for using HOL-generated macros and displays
%% ---------------------------------------------------
\usepackage{holtex}
\usepackage{holtexbasic}
\input{commands}

%% ---------------------------------------------------
%% Package for a table that will break across pages
%% ---------------------------------------------------
\usepackage{longtable}

%% ---------------------------------------------------
%% Packages to bring the contents of other files in
%% ---------------------------------------------------
\usepackage{import}
\usepackage{moreverb}

%% ---------------------------------------------------
%% parameter "language" is set to "ML"
%% ---------------------------------------------------
\lstset{language=ML}

\makeindex

\input{../HOL/HOLReports/HOLcipher}
\input{../HOL/HOLReports/HOLcryptoExercises}

\title{Project 7 Report}
\author{Alfred Murabito}
\date{\today}

\begin{document}

\maketitle{}

\newpage

\begin{abstract}
This report details results for the following exercises from
\textit{Certified Security by Design Using Higher Order Logic}:
15.6.1, 15.6.2, 15.6.3. In these exercises, we 
defined new theorems using the Symmetric/Assymmetric encyrption 
digital signature theorems defined in cipherTheory
\end{abstract}

\newpage

\textbf{Acknowledgments}: I received no assistance with this exercise.

\newpage

\tableofcontents

\newpage

\section{Executive Summary}
\label{sec:executive-summary}



All requirements for this assignment have been satisfied. A description of each exercise is given in the corresponding sections of this report.  These exercises focus on the material presented in Chapter 15 of the textbook, \textit{Certified Security by Design Using Higher Order Logic}.  In addition, pretty-printing was used for appropriate portions of this report.

\newpage

\chapter{Chapter 15}
\label{cha:chapter-15}

\section{Exercise 15.6.1 }
\label{sec:exercise-15.6.1-}

\subsection{Problem Statement}
\label{sec:problem-statement-1}

The following theorems were proved using the symmetric key encyrption theroems. 

\begin{itemize}
  \item \UseVerbatim{HOLcryptoExercisesTheoremsexerciseOneFiveXXSixXXOneaXXthm}
  \item \UseVerbatim{HOLcryptoExercisesTheoremsexerciseOneFiveXXSixXXOnebXXthm}
\end{itemize}


The same theorem \textit{deciphS_one_one} was used to prove both theorems
1a and 1b, using a goal-orientated approach and \textit{ASM_REWRITE_TAC} for 1a and 
\textit{PROVE_TAC} for 1b.

\subsection{Execution Transcript}
\label{sec:execution-transcript-1}

\lstset{frameround=fftt}
\begin{lstlisting}

> val exercise15_6_1a_thm = 
TAC_PROOF(
([], ``! keyk enMsg message.
(deciphS key enMsg = SOME message) <=>
(enMsg = Es key (SOME message))``),
ASM_REWRITE_TAC[deciphS_one_one]
);
# # # # # # <<HOL message: inventing new type variable names: 'a, 'b>>
val exercise15_6_1a_thm =
   |- ∀keyk enMsg message.
     (deciphS key enMsg = SOME message) ⇔
     (enMsg = Es key (SOME message)):
   thm
> 



> val exercise15_6_1b_thm = 
TAC_PROOF(
([],``! keyAlice k text.
(deciphS keyAlice (Es k (SOME text)) =
SOME "This is from Alice") <=>
(k = keyAlice) /\ (text = "This is from Alice")``),
PROVE_TAC [deciphS_one_one]);
# # # # # # Meson search level: ......................
val exercise15_6_1b_thm =
   |- ∀keyAlice k text.
     (deciphS keyAlice (Es k (SOME text)) = SOME "This is from Alice") ⇔
     (k = keyAlice) ∧ (text = "This is from Alice"):
   thm
> 

\end{lstlisting}

\section{Exercise 15.6.2}
\label{sec:exercise-15.6.2}

\subsection{Problem Statement}
\label{sec:problem-statement-2}

Similar to the exercise above, but we prove some theorems for assymmetric encyrption
in exercise 15.6.2. 
\begin{itemize}
  \item \UseVerbatim{HOLcryptoExercisesTheoremsexerciseOneFiveXXSixXXTwoaXXthm}
  \item \UseVerbatim{HOLcryptoExercisesTheoremsexerciseOneFiveXXSixXXTwobXXthm}
\end{itemize}
The theorem used in both prooofs 2a and 2b 
was \textit{deciphP_one_one}

\subsection{Execution Transcript}
\label{sec:execution-transcript-2}

\lstset{frameround=fftt}
\begin{lstlisting}

> val exercise15_6_2a_thm = 
TAC_PROOF(
([], ``! P message.
(deciphP (pubK P) enMsg = SOME message) <=>
(enMsg = Ea (privK P) (SOME message))``),
ASM_REWRITE_TAC [deciphP_one_one]);
# # # # # <<HOL message: inventing new type variable names: 'a, 'b>>
val exercise15_6_2a_thm =
   |- ∀P message.
     (deciphP (pubK P) enMsg = SOME message) ⇔
     (enMsg = Ea (privK P) (SOME message)):
   thm
> 

> val exercise15_6_2b_thm =
TAC_PROOF (
([], ``! key text.
(deciphP (pubK Alice) (Ea key (SOME text)) =
SOME "This is from Alice") <=>
(key = privK Alice) /\ (text = "This is from Alice")``),
PROVE_TAC [deciphP_one_one]
);
# # # # # # # <<HOL message: inventing new type variable names: 'a>>
Meson search level: ......................
val exercise15_6_2b_thm =
   |- ∀key text.
     (deciphP (pubK Alice) (Ea key (SOME text)) =
      SOME "This is from Alice") ⇔
     (key = privK Alice) ∧ (text = "This is from Alice"):
   thm
> 


\end{lstlisting}
\section{Exercise 15.6.3}
\label{sec:exercise-15.6.3}

\subsection{Problem Statement}
\label{sec:problem-statement-3}

The final problem involves a proof using the signature verification theroem 
\textit{signVerify_one_one}.

\UseVerbatim{HOLcryptoExercisesTheoremsexerciseOneFiveXXSixXXThreeXXthm}

\subsection{Execution Transcript}
\label{sec:execution-transcript-3}

\lstset{frameround=fftt}
\begin{lstlisting}
> val exercise15_6_3_thm =
TAC_PROOF (
([], ``! signature.
signVerify (pubK Alice) signature
(SOME "This is from Alice") <=>
(signature =
sign (privK Alice) (hash (SOME "This is from Alice")))``),
PROVE_TAC [signVerify_one_one]
);
# # # # # # # # <<HOL message: inventing new type variable names: 'a>>
Meson search level: ..........
val exercise15_6_3_thm =
   |- ∀signature.
     signVerify (pubK Alice) signature (SOME "This is from Alice") ⇔
     (signature = sign (privK Alice) (hash (SOME "This is from Alice"))):
   thm
> 
\end{lstlisting}
\HOLindex

\appendix{}

\section{Source for cipherScript.sml}
\label{sec:source-ciph}

The following code is from \emph{HOL/cipherScript.sml}
\lstinputlisting{../HOL/cipherScript.sml}

\section{Source for cryptoExercisesScript.sml}
\label{sec:source-crypt}

The following code is from \emph{HOL/cryptoExercisesScript.sml}
\lstinputlisting{../HOL/cryptoExercisesScript.sml}

\end{document}
