
\documentclass[letterpaper]{report}

%% ---------------------------------------------------
%% 634format specifies the format of our reports
%% ---------------------------------------------------
\usepackage{634format}

%% ---------------------------------------------------
%% enumerate 
%% ---------------------------------------------------
\usepackage{enumerate}

%% ---------------------------------------------------
%% listings is used for including our source code in reports
%% ---------------------------------------------------
\usepackage{listings}
\usepackage{textcomp}
\usepackage{indentfirst}

%% ---------------------------------------------------
%% Packages for math environments
%% ---------------------------------------------------
\usepackage{amsmath}

%% ---------------------------------------------------
%% Packages for URLs and hotlinks in the table of contents
%% and symbolic cross references using \ref
%% ---------------------------------------------------
\usepackage{hyperref}

%% ---------------------------------------------------
%% Packages for using HOL-generated macros and displays
%% ---------------------------------------------------
\usepackage{holtex}
\usepackage{holtexbasic}
% =====================================================================
%
% Macros for typesetting the HOL system manual
%
% =====================================================================

% ---------------------------------------------------------------------
% Abbreviations for words and phrases
% ---------------------------------------------------------------------

\newcommand\TUTORIAL{{\footnotesize\sl TUTORIAL}}
\newcommand\DESCRIPTION{{\footnotesize\sl DESCRIPTION}}
\newcommand\REFERENCE{{\footnotesize\sl REFERENCE}}
\newcommand\LOGIC{{\footnotesize\sl LOGIC}}
\newcommand\LIBRARIES{{\footnotesize\sl LIBRARIES}}
\usepackage{textcomp}

\newcommand{\bs}{\texttt{\char'134}} % backslash
\newcommand{\lb}{\texttt{\char'173}} % left brace
\newcommand{\rb}{\texttt{\char'175}} % right brace
\newcommand{\td}{\texttt{\char'176}} % tilde
\newcommand{\lt}{\texttt{\char'74}} % less than
\newcommand{\gt}{\texttt{\char'76}} % greater than
\newcommand{\dol}{\texttt{\char'44}} % dollar
\newcommand{\pipe}{\texttt{\char'174}}
\newcommand{\apost}{\texttt{\textquotesingle}}
% double back quotes ``
\newcommand{\dq}{\texttt{\char'140\char'140}}
%These macros were included by slind:

\newcommand{\holquote}[1]{\dq#1\dq}

\def\HOL{\textsc{Hol}}
\def\holn{\HOL}  % i.e. hol n(inety-eight), no digits in
                 % macro names is a bit of a pain; deciding to do away
                 % with hol98 nomenclature means that we just want to
                 % write HOL for hol98.
\def\holnversion{Kananaskis-11}
\def\holnsversion{Kananaskis~11} % version with space rather than hyphen
\def\LCF{{\small LCF}}
\def\LCFLSM{{\small LCF{\kern-.2em}{\normalsize\_}{\kern0.1em}LSM}}
\def\PPL{{\small PP}{\kern-.095em}$\lambda$}
\def\PPLAMBDA{{\small PPLAMBDA}}
\def\ML{{\small ML}}
\def\holmake{\texttt{Holmake}}

\newcommand\ie{\mbox{\textit{i{.}e{.}}}}
\newcommand\eg{\mbox{\textit{e{.}g{.}}}}
\newcommand\viz{\mbox{viz{.}}}
\newcommand\adhoc{\mbox{\it ad hoc}}
\newcommand\etal{{\it et al.\/}}
% NOTE: \etc produces wrong spacing if used between sentences, that is
% like here \etc End such sentences with non-macro etc.
\newcommand\etc{\mbox{\textit{etc{.}}}}

% ---------------------------------------------------------------------
% Simple abbreviations and macros for mathematical typesetting
% ---------------------------------------------------------------------

\newcommand\fun{{\to}}
\newcommand\prd{{\times}}

\newcommand\conj{\ \wedge\ }
\newcommand\disj{\ \vee\ }
\newcommand\imp{ \Rightarrow }
\newcommand\eqv{\ \equiv\ }
\newcommand\cond{\rightarrow}
\newcommand\vbar{\mid}
\newcommand\turn{\ \vdash\ } % FIXME: "\ " results in extra space
\newcommand\hilbert{\varepsilon}
\newcommand\eqdef{\ \equiv\ }

\newcommand\natnums{\mbox{${\sf N}\!\!\!\!{\sf N}$}}
\newcommand\bools{\mbox{${\sf T}\!\!\!\!{\sf T}$}}

\newcommand\p{$\prime$}
\newcommand\f{$\forall$\ }
\newcommand\e{$\exists$\ }

\newcommand\orr{$\vee$\ }
\newcommand\negg{$\neg$\ }

\newcommand\arrr{$\rightarrow$}
\newcommand\hex{$\sharp $}

\newcommand{\uquant}[1]{\forall #1.\ }
\newcommand{\equant}[1]{\exists #1.\ }
\newcommand{\hquant}[1]{\hilbert #1.\ }
\newcommand{\iquant}[1]{\exists ! #1.\ }
\newcommand{\lquant}[1]{\lambda #1.\ }

\newcommand{\leave}[1]{\\[#1]\noindent}
\newcommand\entails{\mbox{\rule{.3mm}{4mm}\rule[2mm]{.2in}{.3mm}}}

% ---------------------------------------------------------------------
% Font-changing commands
% ---------------------------------------------------------------------

\newcommand{\theory}[1]{\hbox{{\small\tt #1}}}
\newcommand{\theoryimp}[1]{\texttt{#1}}

\newcommand{\con}[1]{{\sf #1}}
\newcommand{\rul}[1]{{\tt #1}}
\newcommand{\ty}[1]{\textsl{#1}}

\newcommand{\ml}[1]{\mbox{{\def\_{\char'137}\texttt{#1}}}}
\newcommand{\holtxt}[1]{\ml{#1}}
\newcommand\ms{\tt}
\newcommand{\s}[1]{{\small #1}}

\newcommand{\pin}[1]{{\bf #1}}
% FIXME: for multichar symbols \mathit should be used.
\def\m#1{\mbox{\normalsize$#1$}}

% ---------------------------------------------------------------------
% Abbreviations for particular mathematical constants etc.
% ---------------------------------------------------------------------

\newcommand\T{\con{T}}
\newcommand\F{\con{F}}
\newcommand\OneOne{\con{One\_One}}
\newcommand\OntoSubset{\con{Onto\_Subset}}
\newcommand\Onto{\con{Onto}}
\newcommand\TyDef{\con{Type\_Definition}}
\newcommand\Inv{\con{Inv}}
\newcommand\com{\con{o}}
\newcommand\Id{\con{I}}
\newcommand\MkPair{\con{Mk\_Pair}}
\newcommand\IsPair{\con{Is\_Pair}}
\newcommand\Fst{\con{Fst}}
\newcommand\Snd{\con{Snd}}
\newcommand\Suc{\con{Suc}}
\newcommand\Nil{\con{Nil}}
\newcommand\Cons{\con{Cons}}
\newcommand\Hd{\con{Hd}}
\newcommand\Tl{\con{Tl}}
\newcommand\Null{\con{Null}}
\newcommand\ListPrimRec{\con{List\_Prim\_Rec}}


\newcommand\SimpRec{\con{Simp\_Rec}}
\newcommand\SimpRecRel{\con{Simp\_Rec\_Rel}}
\newcommand\SimpRecFun{\con{Simp\_Rec\_Fun}}
\newcommand\PrimRec{\con{Prim\_Rec}}
\newcommand\PrimRecRel{\con{Prim\_Rec\_Rel}}
\newcommand\PrimRecFun{\con{Prim\_Rec\_Fun}}

\newcommand\bool{\ty{bool}}
\newcommand\num{\ty{num}}
\newcommand\ind{\ty{ind}}
\newcommand\lst{\ty{list}}

% ---------------------------------------------------------------------
% \minipagewidth = \textwidth minus 1.02 em
% ---------------------------------------------------------------------

\newlength{\minipagewidth}
\setlength{\minipagewidth}{\textwidth}
\addtolength{\minipagewidth}{-1.02em}

% ---------------------------------------------------------------------
% Environment for the items on the title page of a case study
% ---------------------------------------------------------------------

\newenvironment{inset}[1]{\noindent{\large\bf #1}\begin{list}%
{}{\setlength{\leftmargin}{\parindent}%
\setlength{\topsep}{-.1in}}\item }{\end{list}\vskip .4in}

% ---------------------------------------------------------------------
% Macros for little HOL sessions displayed in boxes.
%
% Usage: (1) \setcounter{sessioncount}{1} resets the session counter
%
%        (2) \begin{session}\begin{verbatim}
%             .
%              < lines from hol session >
%             .
%            \end{verbatim}\end{session}
%
%            typesets the session in a numbered box.
% ---------------------------------------------------------------------

\newlength{\hsbw}
\setlength{\hsbw}{\textwidth}
\addtolength{\hsbw}{-\arrayrulewidth}
\addtolength{\hsbw}{-\tabcolsep}
\newcommand\HOLSpacing{13pt}

\newcounter{sessioncount}
\setcounter{sessioncount}{0}

\newenvironment{session}{\begin{flushleft}
 \refstepcounter{sessioncount}
 \begin{tabular}{@{}|c@{}|@{}}\hline
 \begin{minipage}[b]{\hsbw}
 \vspace*{-.5pt}
 \begin{flushright}
 \rule{0.01in}{.15in}\rule{0.3in}{0.01in}\hspace{-0.35in}
 \raisebox{0.04in}{\makebox[0.3in][c]{\footnotesize\sl \thesessioncount}}
 \end{flushright}
 \vspace*{-.55in}
 \begingroup\small\baselineskip\HOLSpacing}{\endgroup\end{minipage}\\ \hline
 \end{tabular}
 \end{flushleft}}

% ---------------------------------------------------------------------
% Macro for boxed ML functions, etc.
%
% Usage: (1) \begin{holboxed}\begin{verbatim}
%               .
%               < lines giving names and types of mk functions >
%               .
%            \end{verbatim}\end{holboxed}
%
%            typesets the given lines in a box.
%
%            Conventions: lines are left-aligned under the "g" of begin,
%            and used to highlight primary reference for the ml function(s)
%            that appear in the box.
% ---------------------------------------------------------------------

\newenvironment{holboxed}{\begin{flushleft}
  \begin{tabular}{@{}|c@{}|@{}}\hline
  \begin{minipage}[b]{\hsbw}
% \vspace*{-.55in}
  \vspace*{.06in}
  \begingroup\small\baselineskip\HOLSpacing}{\endgroup\end{minipage}\\ \hline
  \end{tabular}
  \end{flushleft}}

% ---------------------------------------------------------------------
% Macro for unboxed ML functions, etc.
%
% Usage: (1) \begin{hol}\begin{verbatim}
%               .
%               < lines giving names and types of mk functions >
%               .
%            \end{verbatim}\end{hol}
%
%            typesets the given lines exactly like {boxed}, except there's
%            no box.
%
%            Conventions: lines are left-aligned under the "g" of begin,
%            and used to display ML code in verbatim, left aligned.
% ---------------------------------------------------------------------

\newenvironment{hol}{\begin{flushleft}
 \begin{tabular}{c@{}@{}}
 \begin{minipage}[b]{\hsbw}
% \vspace*{-.55in}
 \vspace*{.06in}
 \begingroup\small\baselineskip\HOLSpacing}{\endgroup\end{minipage}\\
 \end{tabular}
 \end{flushleft}}

% ---------------------------------------------------------------------
% Emphatic brackets
% ---------------------------------------------------------------------

\newcommand\leb{\lbrack\!\lbrack}
\newcommand\reb{\rbrack\!\rbrack}


% ---------------------------------------------------------------------
% Quotations
% ---------------------------------------------------------------------


%These macros were included by ap; they are used in Chapters 9 and 10
%of the HOL DESCRIPTION

\newcommand{\inds}%standard infinite set
 {\mbox{\rm I}}

\newcommand{\ch}%standard choice function
 {\mbox{\rm ch}}

\newcommand{\den}[1]%denotational brackets
 {[\![#1]\!]}

\newcommand{\two}%standard 2-element set
 {\mbox{\rm 2}}


%% ---------------------------------------------------
%% Package for a table that will break across pages
%% ---------------------------------------------------
\usepackage{longtable}

%% ---------------------------------------------------
%% Packages to bring the contents of other files in
%% ---------------------------------------------------
\usepackage{import}
\usepackage{moreverb}

%% ---------------------------------------------------
%% parameter "language" is set to "ML"
%% ---------------------------------------------------
\lstset{language=ML}



\title{Project 2}
\author{\textbf{Alfred Murabito}}
\date{\today}

\begin{document}
\maketitle{}

\begin{abstract}
\noindent{}This report details results for the following exercise from \textit{Certified Security by Design Using Higher Order Logic}: 4.6.3, 4.6.4, 5.3.4, 5.3.5, and 6.2.1.  
We define functions using pattern matching and got famaliar with HOL's logical symbols.
\end{abstract}

\newpage

\textbf{Acknowledgments}: I received no assistance with this project.

\newpage

\tableofcontents

\newpage

\chapter{Executive Summary}
\label{sec:executive-summary}

All requirements for this project have been satisfied.  A description and the results of each exercise are detailed in this project report.  

\newpage

\chapter{Exercise 4.6.3}
\label{sec:ex-4-6-3}

\section{Problem Statement}

In this exercise I defined functions using the both the fun keyword and by using val assignmetns
on lambda functions. Test results verifying the functions' intended behaviors follows.

\begin{description}
\item[Capabilities:] ML functions defined to perform the following tasks:
   \begin{itemize}
   \item return the sum of elements in a 3-tuple
   \item determine whether one integer value is less than another
   \item concatenate two strings
   \item append lists
   \item determine whether one integer value is greater than another and return the larger integer value
   \end{itemize}
\end{description}


\section{Relevant Code}
\label{sec:problem-statement}

The following code snippets illustrate relevant sections of code developed to perform the above described tasks.\newline 

Part A:
\begin{lstlisting}
val funa1 = fn (x, y, z) => x + y + z;
fun funa2 (x, y, z) = x + y + z;
\end{lstlisting}

Part B:
\begin{lstlisting}
val funb1 = (fn x => (fn y => x < y)); 
fun funb2 x y = x < y;
\end{lstlisting}

Part C:
\begin{lstlisting}
val func1 = fn s1 => (fn s2 => s1 ^ s2);
fun func2 s1 s2 = s1 ^ s2;
\end{lstlisting}

Part D:
\begin{lstlisting}
val fund1 = fn list1 => (fn list2 => list1 @ list2);
fun fund2 list1 list2 = list1 @ list2;
\end{lstlisting}

Part E:
\begin{lstlisting}
val fune1 = fn (x,y) => if x > y then x else y;
fun fune2 (x,y) = if x > y then x else y;
\end{lstlisting}


\section{Test Cases and Results}
\label{sec:tests}

\vspace{12px}
\indent Test cases for Part A:
\begin{lstlisting}
funa1 (1,3,4);             (* Expect 8 *)
funa2 (1,3,4);
funa1 (~1,0,1);            (* Expect 0 *)
funa2 (~1,0,1);
funa1 (489,2,34);          (* Expect 525 *)
funa2 (489,2,34);
\end{lstlisting}

\verbatiminput{ML/ex463a.trans}

Test cases for Part B:
\begin{lstlisting}
funb1 1 2;                 (* Expect true *)
funb2 1 2;
funb1 ~1 0;                (* Expect true *)
funb2 ~1 0;
funb1 ~3 ~44;              (* Expect false *)
funb2 ~3 ~44;
\end{lstlisting}

Results for Part B:
\verbatiminput{ML/ex463b.trans}

Test cases for Part C:
\begin{lstlisting}
func1 "hello" " world";      (* Expect "hello world" *)
func2 "hello" " world";
func1 "" "asdf";            (* Expect "asdf" *)
func2 "" "asdf";
\end{lstlisting}

Results for Part C:
\verbatiminput{ML/ex463c.trans}

Test cases for Part D:
\begin{lstlisting}
fund1 [1,2,3] [4,5];       (* Expect [1,2,3,4,5] *)
fund2 [1,2,3] [4,5];
fund1 [] [1,2,3];          (* Expect [1,2,3] *)
fund2 [] [1,2,3];
fund1 ["a","b","c"] [];          (* Expect ["a","b","c"] *)
fund2 ["a","b","c"] [];          (* Expect ["a","b","c"] *)
fund1 [true, false] [false, false, false]  (* Expect [true, false, false, false] *)
fund2 [true, false] [false, false, false]  (* Expect [true, false, false, false] *)
\end{lstlisting}

Results for Part D:
\verbatiminput{ML/ex463d.trans}

Test cases for Part E:
\begin{lstlisting}
fune1 (34, 55);	                 (* Expect 55 *)
fune2 (34, 55);
fune1 (~11, 478);	         (* Expect 478 *)
fune2 (~11, 478);
fune1 (0, 0);                    (* Expect 0 *)
fune2 (0, 0);               
\end{lstlisting}

Results for Part E:
\verbatiminput{ML/ex463e.trans}

\newpage


\chapter{Exercise 4.6.4}
\label{sec:ex-4-6-4}

\section{Problem Statement}
For this exercise a function is defined that squares each element of a list, unless the list is empty. 
We use a ``let'' expression to define the function

\section{Relevant Code}

\begin{lstlisting}
fun listSquares [] = []
  | listSquares (x::xs) = 
  let
    fun square y = y * y
  in
    square(x)::listSquares(xs)
  end;
\end{lstlisting}

\section{Test Cases and Results}

Test cases for Exercise 4.6.4:
\begin{lstlisting}
listSquares [];               (* Expect [] *)
listSquares [1,2,4,8];        (* Expect [1,4,16,64] *)
listSquares [5,12,7,~1];      (* Expect [25,144,49,1] *)
\end{lstlisting}

Results for Exercise 4.6.4:
\verbatiminput{ML/ex464.trans}

\newpage


\chapter{Exercise 5.3.4}
\label{sec:ex-5-3-4}

\section{Problem Statement}
The goal of this exercise is to define the function Filter that has identical behavior to the filter function.
Filter accepts a function B and a list where every element ``x'' in the list where ``B x'' is true is in the list
returned. Test cases matching the ML function and our function follows as well.

\section{Relevant Code}

\begin{lstlisting}
fun Filter B [] = []
  | Filter B (x::xs) = if B x then x :: (Filter B xs) else (Filter B xs);
\end{lstlisting}

\section{Test Cases and Results}

Test cases for Exercise 5.3.4:
\begin{lstlisting}
Filter (fn x => x < 5) [];		   (* Expect [] *)
filter (fn x => x < 5) [];

Filter (fn x => x < 5) [4,6];		   (* Expect [4] *)
filter (fn x => x < 5) [4,6];

Filter (fn x => x < 5) [1,2,4,8,16];	   (* Expect [1,2,4] *)
filter (fn x => x < 5) [1,2,4,8,16];

Filter (fn x => x < 5) [5,6,7,8];	   (* Expect [] *)
filter (fn x => x < 5) [5,6,7,8];

Filter (fn x => x = "cat") ["dog","bird"]; (* Expect [] *)
filter (fn x => x = "cat") ["dog","bird"];

Filter (fn x => x = "cat") ["cat","dog","cat"];	     (* Expect ["cat","cat"] *)
filter (fn x => x = "cat") ["cat","dog","cat"];
\end{lstlisting}

Results for Exercise 5.3.4:

\verbatiminput{ML/ex534.trans}

\newpage



\chapter{Exercise 5.3.5}
\label{sec:ex-5-3-5}

\section{Problem Statement}
This exercise involves defining a function which accepts a number ``n'' and  a list of pairs of integers
and returns a list with the sum of all pairs that were both greter than the given parameter ``n''. 

\section{Relevant Code}

\begin{lstlisting}
fun addPairsGreaterThan n list =
let
  fun addPair (x,y) = x + y
  val filteredList = filter (fn (x,y) => x > n andalso y > n) list
in
  map addPair filteredList
end;
\end{lstlisting}

\section{Test Cases and Results}

Test cases for Exercise 5.3.5:
\begin{lstlisting}
(* Test Cases                   *)
addPairsGreaterThan 0 [(0,1),(2,0),(2,3),(4,5)];   (* expect [5,9] *)
addPairsGreaterThan 2 [(0,1),(2,0),(2,3),(4,5)];   (* expect [9] *)
addPairsGreaterThan 4 [(0,1),(2,0),(2,3),(4,5)];   (* expect [] *)
\end{lstlisting}

Results for Exercise 5.3.5:

\verbatiminput{ML/ex535.trans}

\newpage


\chapter{Exercise 6.2.1}
\label{sec:ex-6-2-1}

\section{Problem Statement}
For this exercise HOL functionality is implemented and tested for basic HOL types and operators.

\section{Relevant Code}

\begin{lstlisting}
(* 1: Enter the HOL equivalent of P(x) *)
``P x ==> Q y``;

(* 2: Constrain x to HOL type :num and y to :bool *)
``P (x:num) ==> Q (y:bool)``;

(* 3: Enter the HOL equivalent of forall x y.P(x) ==> Q(y) *)
``!x y.P(x) ==> Q(y)``;

(* 4: Enter the HOL equivalent of \exists (x:num).R(x:'a) *)
``?(x:num).R(x:num)``;

(* 5: *)
``~!x.P(x)\/Q(x) = ?x.~P(x)/\~Q(x)``;

(* 6: All people are mortal, where P(x) means x is a person, *)
(*   and M(x) means x is mortal. *)
``!x.P(x) ==> M(x)``;

(* 7: Some people are funny. *)
``?x.Funny(x)``;

(* 4b: Enter the HOL equivalent of \exists (x:num).R(x:'a) *)
``?(x:num).R(x:'a)``;
\end{lstlisting}

\section{Test Cases and Results}

For Exercise 6.2.1 the test cases are the HOL types and operators shown in the ``Relevant Code'' section above.  

\verbatiminput{ML/ex621.trans}

\begin{description}
\item[Results] Below are the observations for each test case
  \begin{enumerate}
  \item x is type 'a, y is type 'b', P is type ('a -> bool), Q is type ('b -> bool)
  \item x is type ``num'' and y is type ``bool'' as expected
  \item x and y are polymorphic types 'a, 'b, since none specified. P and Q must return bools since they are part of the implication statement
  \item The HOL term fails since x is defined as type ``num'' then again as 'a. The correct statement with x as ``num'' is shown with the rest of the correct ones.
  \item Entered as described and executed
  \item M(x) represents subject is mortal, P(x) represents subject is a person
  \item Funny(x) represents the subject is funny.
  \end{enumerate}
\end{description}


\newpage



\appendix

\chapter{Exercise 4.6.3 Source Code}

\verbatiminput{ML/ex463.sml}

\newpage

\chapter{Exercise 4.6.4 Source Code}

\verbatiminput{ML/ex464.sml}

\newpage

\chapter{Exercise 5.3.4 Source Code}

\verbatiminput{ML/ex534.sml}

\newpage

\chapter{Exercise 5.3.5 Source Code}

\verbatiminput{ML/ex535.sml}

\newpage

\chapter{Exercise 6.2.1 Source Code}

\verbatiminput{ML/ex621.sml}


\end{document}
