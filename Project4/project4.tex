
\documentclass[11pt, twoside]{article}

%% ---------------------------------------------------
%% 634format specifies the format of our reports
%% ---------------------------------------------------
\usepackage{634format}

%% ---------------------------------------------------
%% enumerate 
%% ---------------------------------------------------
\usepackage{enumerate}

%% ---------------------------------------------------
%% listings is used for including our source code in reports
%% ---------------------------------------------------
\usepackage{listings}
\usepackage{textcomp}
\usepackage{indentfirst}

%% ---------------------------------------------------
%% Packages for math environments
%% ---------------------------------------------------
\usepackage{amsmath}

%% ---------------------------------------------------
%% Packages for URLs and hotlinks in the table of contents
%% and symbolic cross references using \ref
%% ---------------------------------------------------
\usepackage{hyperref}

%% ---------------------------------------------------
%% Packages for using HOL-generated macros and displays
%% ---------------------------------------------------
\usepackage{holtex}
\usepackage{holtexbasic}
% =====================================================================
%
% Macros for typesetting the HOL system manual
%
% =====================================================================

% ---------------------------------------------------------------------
% Abbreviations for words and phrases
% ---------------------------------------------------------------------

\newcommand\TUTORIAL{{\footnotesize\sl TUTORIAL}}
\newcommand\DESCRIPTION{{\footnotesize\sl DESCRIPTION}}
\newcommand\REFERENCE{{\footnotesize\sl REFERENCE}}
\newcommand\LOGIC{{\footnotesize\sl LOGIC}}
\newcommand\LIBRARIES{{\footnotesize\sl LIBRARIES}}
\usepackage{textcomp}

\newcommand{\bs}{\texttt{\char'134}} % backslash
\newcommand{\lb}{\texttt{\char'173}} % left brace
\newcommand{\rb}{\texttt{\char'175}} % right brace
\newcommand{\td}{\texttt{\char'176}} % tilde
\newcommand{\lt}{\texttt{\char'74}} % less than
\newcommand{\gt}{\texttt{\char'76}} % greater than
\newcommand{\dol}{\texttt{\char'44}} % dollar
\newcommand{\pipe}{\texttt{\char'174}}
\newcommand{\apost}{\texttt{\textquotesingle}}
% double back quotes ``
\newcommand{\dq}{\texttt{\char'140\char'140}}
%These macros were included by slind:

\newcommand{\holquote}[1]{\dq#1\dq}

\def\HOL{\textsc{Hol}}
\def\holn{\HOL}  % i.e. hol n(inety-eight), no digits in
                 % macro names is a bit of a pain; deciding to do away
                 % with hol98 nomenclature means that we just want to
                 % write HOL for hol98.
\def\holnversion{Kananaskis-11}
\def\holnsversion{Kananaskis~11} % version with space rather than hyphen
\def\LCF{{\small LCF}}
\def\LCFLSM{{\small LCF{\kern-.2em}{\normalsize\_}{\kern0.1em}LSM}}
\def\PPL{{\small PP}{\kern-.095em}$\lambda$}
\def\PPLAMBDA{{\small PPLAMBDA}}
\def\ML{{\small ML}}
\def\holmake{\texttt{Holmake}}

\newcommand\ie{\mbox{\textit{i{.}e{.}}}}
\newcommand\eg{\mbox{\textit{e{.}g{.}}}}
\newcommand\viz{\mbox{viz{.}}}
\newcommand\adhoc{\mbox{\it ad hoc}}
\newcommand\etal{{\it et al.\/}}
% NOTE: \etc produces wrong spacing if used between sentences, that is
% like here \etc End such sentences with non-macro etc.
\newcommand\etc{\mbox{\textit{etc{.}}}}

% ---------------------------------------------------------------------
% Simple abbreviations and macros for mathematical typesetting
% ---------------------------------------------------------------------

\newcommand\fun{{\to}}
\newcommand\prd{{\times}}

\newcommand\conj{\ \wedge\ }
\newcommand\disj{\ \vee\ }
\newcommand\imp{ \Rightarrow }
\newcommand\eqv{\ \equiv\ }
\newcommand\cond{\rightarrow}
\newcommand\vbar{\mid}
\newcommand\turn{\ \vdash\ } % FIXME: "\ " results in extra space
\newcommand\hilbert{\varepsilon}
\newcommand\eqdef{\ \equiv\ }

\newcommand\natnums{\mbox{${\sf N}\!\!\!\!{\sf N}$}}
\newcommand\bools{\mbox{${\sf T}\!\!\!\!{\sf T}$}}

\newcommand\p{$\prime$}
\newcommand\f{$\forall$\ }
\newcommand\e{$\exists$\ }

\newcommand\orr{$\vee$\ }
\newcommand\negg{$\neg$\ }

\newcommand\arrr{$\rightarrow$}
\newcommand\hex{$\sharp $}

\newcommand{\uquant}[1]{\forall #1.\ }
\newcommand{\equant}[1]{\exists #1.\ }
\newcommand{\hquant}[1]{\hilbert #1.\ }
\newcommand{\iquant}[1]{\exists ! #1.\ }
\newcommand{\lquant}[1]{\lambda #1.\ }

\newcommand{\leave}[1]{\\[#1]\noindent}
\newcommand\entails{\mbox{\rule{.3mm}{4mm}\rule[2mm]{.2in}{.3mm}}}

% ---------------------------------------------------------------------
% Font-changing commands
% ---------------------------------------------------------------------

\newcommand{\theory}[1]{\hbox{{\small\tt #1}}}
\newcommand{\theoryimp}[1]{\texttt{#1}}

\newcommand{\con}[1]{{\sf #1}}
\newcommand{\rul}[1]{{\tt #1}}
\newcommand{\ty}[1]{\textsl{#1}}

\newcommand{\ml}[1]{\mbox{{\def\_{\char'137}\texttt{#1}}}}
\newcommand{\holtxt}[1]{\ml{#1}}
\newcommand\ms{\tt}
\newcommand{\s}[1]{{\small #1}}

\newcommand{\pin}[1]{{\bf #1}}
% FIXME: for multichar symbols \mathit should be used.
\def\m#1{\mbox{\normalsize$#1$}}

% ---------------------------------------------------------------------
% Abbreviations for particular mathematical constants etc.
% ---------------------------------------------------------------------

\newcommand\T{\con{T}}
\newcommand\F{\con{F}}
\newcommand\OneOne{\con{One\_One}}
\newcommand\OntoSubset{\con{Onto\_Subset}}
\newcommand\Onto{\con{Onto}}
\newcommand\TyDef{\con{Type\_Definition}}
\newcommand\Inv{\con{Inv}}
\newcommand\com{\con{o}}
\newcommand\Id{\con{I}}
\newcommand\MkPair{\con{Mk\_Pair}}
\newcommand\IsPair{\con{Is\_Pair}}
\newcommand\Fst{\con{Fst}}
\newcommand\Snd{\con{Snd}}
\newcommand\Suc{\con{Suc}}
\newcommand\Nil{\con{Nil}}
\newcommand\Cons{\con{Cons}}
\newcommand\Hd{\con{Hd}}
\newcommand\Tl{\con{Tl}}
\newcommand\Null{\con{Null}}
\newcommand\ListPrimRec{\con{List\_Prim\_Rec}}


\newcommand\SimpRec{\con{Simp\_Rec}}
\newcommand\SimpRecRel{\con{Simp\_Rec\_Rel}}
\newcommand\SimpRecFun{\con{Simp\_Rec\_Fun}}
\newcommand\PrimRec{\con{Prim\_Rec}}
\newcommand\PrimRecRel{\con{Prim\_Rec\_Rel}}
\newcommand\PrimRecFun{\con{Prim\_Rec\_Fun}}

\newcommand\bool{\ty{bool}}
\newcommand\num{\ty{num}}
\newcommand\ind{\ty{ind}}
\newcommand\lst{\ty{list}}

% ---------------------------------------------------------------------
% \minipagewidth = \textwidth minus 1.02 em
% ---------------------------------------------------------------------

\newlength{\minipagewidth}
\setlength{\minipagewidth}{\textwidth}
\addtolength{\minipagewidth}{-1.02em}

% ---------------------------------------------------------------------
% Environment for the items on the title page of a case study
% ---------------------------------------------------------------------

\newenvironment{inset}[1]{\noindent{\large\bf #1}\begin{list}%
{}{\setlength{\leftmargin}{\parindent}%
\setlength{\topsep}{-.1in}}\item }{\end{list}\vskip .4in}

% ---------------------------------------------------------------------
% Macros for little HOL sessions displayed in boxes.
%
% Usage: (1) \setcounter{sessioncount}{1} resets the session counter
%
%        (2) \begin{session}\begin{verbatim}
%             .
%              < lines from hol session >
%             .
%            \end{verbatim}\end{session}
%
%            typesets the session in a numbered box.
% ---------------------------------------------------------------------

\newlength{\hsbw}
\setlength{\hsbw}{\textwidth}
\addtolength{\hsbw}{-\arrayrulewidth}
\addtolength{\hsbw}{-\tabcolsep}
\newcommand\HOLSpacing{13pt}

\newcounter{sessioncount}
\setcounter{sessioncount}{0}

\newenvironment{session}{\begin{flushleft}
 \refstepcounter{sessioncount}
 \begin{tabular}{@{}|c@{}|@{}}\hline
 \begin{minipage}[b]{\hsbw}
 \vspace*{-.5pt}
 \begin{flushright}
 \rule{0.01in}{.15in}\rule{0.3in}{0.01in}\hspace{-0.35in}
 \raisebox{0.04in}{\makebox[0.3in][c]{\footnotesize\sl \thesessioncount}}
 \end{flushright}
 \vspace*{-.55in}
 \begingroup\small\baselineskip\HOLSpacing}{\endgroup\end{minipage}\\ \hline
 \end{tabular}
 \end{flushleft}}

% ---------------------------------------------------------------------
% Macro for boxed ML functions, etc.
%
% Usage: (1) \begin{holboxed}\begin{verbatim}
%               .
%               < lines giving names and types of mk functions >
%               .
%            \end{verbatim}\end{holboxed}
%
%            typesets the given lines in a box.
%
%            Conventions: lines are left-aligned under the "g" of begin,
%            and used to highlight primary reference for the ml function(s)
%            that appear in the box.
% ---------------------------------------------------------------------

\newenvironment{holboxed}{\begin{flushleft}
  \begin{tabular}{@{}|c@{}|@{}}\hline
  \begin{minipage}[b]{\hsbw}
% \vspace*{-.55in}
  \vspace*{.06in}
  \begingroup\small\baselineskip\HOLSpacing}{\endgroup\end{minipage}\\ \hline
  \end{tabular}
  \end{flushleft}}

% ---------------------------------------------------------------------
% Macro for unboxed ML functions, etc.
%
% Usage: (1) \begin{hol}\begin{verbatim}
%               .
%               < lines giving names and types of mk functions >
%               .
%            \end{verbatim}\end{hol}
%
%            typesets the given lines exactly like {boxed}, except there's
%            no box.
%
%            Conventions: lines are left-aligned under the "g" of begin,
%            and used to display ML code in verbatim, left aligned.
% ---------------------------------------------------------------------

\newenvironment{hol}{\begin{flushleft}
 \begin{tabular}{c@{}@{}}
 \begin{minipage}[b]{\hsbw}
% \vspace*{-.55in}
 \vspace*{.06in}
 \begingroup\small\baselineskip\HOLSpacing}{\endgroup\end{minipage}\\
 \end{tabular}
 \end{flushleft}}

% ---------------------------------------------------------------------
% Emphatic brackets
% ---------------------------------------------------------------------

\newcommand\leb{\lbrack\!\lbrack}
\newcommand\reb{\rbrack\!\rbrack}


% ---------------------------------------------------------------------
% Quotations
% ---------------------------------------------------------------------


%These macros were included by ap; they are used in Chapters 9 and 10
%of the HOL DESCRIPTION

\newcommand{\inds}%standard infinite set
 {\mbox{\rm I}}

\newcommand{\ch}%standard choice function
 {\mbox{\rm ch}}

\newcommand{\den}[1]%denotational brackets
 {[\![#1]\!]}

\newcommand{\two}%standard 2-element set
 {\mbox{\rm 2}}


%% ---------------------------------------------------
%% Package for a table that will break across pages
%% ---------------------------------------------------
\usepackage{longtable}

%% ---------------------------------------------------
%% Packages to bring the contents of other files in
%% ---------------------------------------------------
\usepackage{import}
\usepackage{moreverb}

%% ---------------------------------------------------
%% parameter "language" is set to "ML"
%% ---------------------------------------------------
\lstset{language=ML}

\makeindex


\title{Project 4}
\author{\textbf{Alfred Murabito}}
\date{\today}

\begin{document}
\maketitle{}

\newpage

\begin{abstract}
   \noindent{}This report details results for the following exercises from \textit{Certified Security by Design Using Higher Order Logic}: 9.5.1, 9.5.2, 9.5.3, 10.4.1, 10.4.2, and 
10.4.3.  Chapter 9 of the textbook focuses on goal-oriented proofs while  Chapter 10 focuses on dealing with assumptions in goal-oriented proofs using PAT_ASSUM to match our supplied
terms to those in the assumptions..  Each of the exercises includes a problem statement, relevant code, execution transcripts, and our pretty-printed theorems. 
\end{abstract}

\newpage

\textbf{Acknowledgments}: I received no assistance with this project.

\newpage

\tableofcontents

\newpage

\section{Executive Summary}
\label{sec:executive-summary}

All requirements for this project have been satisfied.  A description of each exercise from Chapter 9 and 10 are included in the following chapters.  
Our HOL script files are contained in the HOL directory and pretty-print our theories to reports using EmiTex in the HOLReports directory. In the appendix
we also include all HOL source files.

\newpage


\section{Chapter 9 Exercises}
\label{sec:Chap9}

\subsection{Problem Statement}
Exercises 9.5.1 and 9.5.2 involve proving the absorption and constructiveDilemma theorems respectively. In exercise 9.5.3, we prove both theorems again using PROVE_TAC.

\subsection{Relevant Code}

Exercise 9.5.1
\begin{lstlisting}[basicstyle=\small,]
  val absorptionRule = 
TAC_PROOF (
( [], ``!(p:bool)(q:bool).(p ==> q) ==> p ==> (p /\ q)``),
REPEAT STRIP_TAC THENL
[(ACCEPT_TAC (ASSUME ``p:bool``)),
(ACCEPT_TAC (MP (ASSUME ``p:bool==>q:bool``) (ASSUME ``p:bool``)))])
\end{lstlisting}

Exercise 9.5.2
\begin{lstlisting}[basicstyle=\small,]
val constructiveDilemmaRule = 
TAC_PROOF (
( [], ``!(p:bool)(q:bool)(r:bool)(s:bool).(p==>q)/\(r==>s)==>(p\/r)==>(q\/s)``),
(REPEAT STRIP_TAC THENL
[(RES_TAC THEN (ACCEPT_TAC (DISJ1 (ASSUME ``q:bool``) ``s:bool``))),
(RES_TAC THEN (ACCEPT_TAC (DISJ2 ``q:bool`` (ASSUME ``s:bool``))))])
);
\end{lstlisting}

Exercise 9.5.3
\begin{lstlisting}[basicstyle=\small,]
val absorptionRule2 = 
TAC_PROOF (
( [], ``!(p:bool)(q:bool).(p ==> q) ==> p ==> (p /\ q)``),
PROVE_TAC [])

val constructiveDilemmaRule2 = 
TAC_PROOF (
( [], ``!(p:bool)(q:bool)(r:bool)(s:bool).(p==>q) /\ (r==>s) ==> p ==> (p /\ q)``),
PROVE_TAC [])
\end{lstlisting}

\newpage 

\subsection{Test Results}


Exercise 9.5.1:
\verbatiminput{HOL/ex951.trans}


Exercise 9.5.2:
\verbatiminput{HOL/ex952.trans}


Exercise 9.5.3:
\verbatiminput{HOL/ex953.trans}



\newcommand{\HOLexerciseNineDate}{22 March 2020}
\newcommand{\HOLexerciseNineTime}{15:39}
\begin{SaveVerbatim}{HOLexerciseNineTheoremsabsorptionRule}
\HOLTokenTurnstile{} \HOLSymConst{\HOLTokenForall{}}\HOLBoundVar{p} \HOLBoundVar{q}. (\HOLBoundVar{p} \HOLSymConst{\HOLTokenImp{}} \HOLBoundVar{q}) \HOLSymConst{\HOLTokenImp{}} \HOLBoundVar{p} \HOLSymConst{\HOLTokenImp{}} \HOLBoundVar{p} \HOLSymConst{\HOLTokenConj{}} \HOLBoundVar{q}
\end{SaveVerbatim}
\newcommand{\HOLexerciseNineTheoremsabsorptionRule}{\UseVerbatim{HOLexerciseNineTheoremsabsorptionRule}}
\begin{SaveVerbatim}{HOLexerciseNineTheoremsabsorptionRuleTwo}
\HOLTokenTurnstile{} \HOLSymConst{\HOLTokenForall{}}\HOLBoundVar{p} \HOLBoundVar{q}. (\HOLBoundVar{p} \HOLSymConst{\HOLTokenImp{}} \HOLBoundVar{q}) \HOLSymConst{\HOLTokenImp{}} \HOLBoundVar{p} \HOLSymConst{\HOLTokenImp{}} \HOLBoundVar{p} \HOLSymConst{\HOLTokenConj{}} \HOLBoundVar{q}
\end{SaveVerbatim}
\newcommand{\HOLexerciseNineTheoremsabsorptionRuleTwo}{\UseVerbatim{HOLexerciseNineTheoremsabsorptionRuleTwo}}
\begin{SaveVerbatim}{HOLexerciseNineTheoremsconstructiveDilemmaRule}
\HOLTokenTurnstile{} \HOLSymConst{\HOLTokenForall{}}\HOLBoundVar{p} \HOLBoundVar{q} \HOLBoundVar{r} \HOLBoundVar{s}. (\HOLBoundVar{p} \HOLSymConst{\HOLTokenImp{}} \HOLBoundVar{q}) \HOLSymConst{\HOLTokenConj{}} (\HOLBoundVar{r} \HOLSymConst{\HOLTokenImp{}} \HOLBoundVar{s}) \HOLSymConst{\HOLTokenImp{}} \HOLBoundVar{p} \HOLSymConst{\HOLTokenDisj{}} \HOLBoundVar{r} \HOLSymConst{\HOLTokenImp{}} \HOLBoundVar{q} \HOLSymConst{\HOLTokenDisj{}} \HOLBoundVar{s}
\end{SaveVerbatim}
\newcommand{\HOLexerciseNineTheoremsconstructiveDilemmaRule}{\UseVerbatim{HOLexerciseNineTheoremsconstructiveDilemmaRule}}
\begin{SaveVerbatim}{HOLexerciseNineTheoremsconstructiveDilemmaRuleTwo}
\HOLTokenTurnstile{} \HOLSymConst{\HOLTokenForall{}}\HOLBoundVar{p} \HOLBoundVar{q} \HOLBoundVar{r} \HOLBoundVar{s}. (\HOLBoundVar{p} \HOLSymConst{\HOLTokenImp{}} \HOLBoundVar{q}) \HOLSymConst{\HOLTokenConj{}} (\HOLBoundVar{r} \HOLSymConst{\HOLTokenImp{}} \HOLBoundVar{s}) \HOLSymConst{\HOLTokenImp{}} \HOLBoundVar{p} \HOLSymConst{\HOLTokenImp{}} \HOLBoundVar{p} \HOLSymConst{\HOLTokenConj{}} \HOLBoundVar{q}
\end{SaveVerbatim}
\newcommand{\HOLexerciseNineTheoremsconstructiveDilemmaRuleTwo}{\UseVerbatim{HOLexerciseNineTheoremsconstructiveDilemmaRuleTwo}}
\newcommand{\HOLexerciseNineTheorems}{
\HOLThmTag{exercise9}{absorptionRule}\HOLexerciseNineTheoremsabsorptionRule
\HOLThmTag{exercise9}{absorptionRule2}\HOLexerciseNineTheoremsabsorptionRuleTwo
\HOLThmTag{exercise9}{constructiveDilemmaRule}\HOLexerciseNineTheoremsconstructiveDilemmaRule
\HOLThmTag{exercise9}{constructiveDilemmaRule2}\HOLexerciseNineTheoremsconstructiveDilemmaRuleTwo
}

\newcommand{\HOLexerciseOneZeroDate}{22 March 2020}
\newcommand{\HOLexerciseOneZeroTime}{15:45}
\begin{SaveVerbatim}{HOLexerciseOneZeroTheoremsproblemOneXXthm}
\HOLTokenTurnstile{} \HOLFreeVar{M} \HOLFreeVar{s}
\end{SaveVerbatim}
\newcommand{\HOLexerciseOneZeroTheoremsproblemOneXXthm}{\UseVerbatim{HOLexerciseOneZeroTheoremsproblemOneXXthm}}
\begin{SaveVerbatim}{HOLexerciseOneZeroTheoremsproblemTwoXXthm}
\HOLTokenTurnstile{} \HOLFreeVar{p} \HOLSymConst{\HOLTokenImp{}} \HOLSymConst{\HOLTokenNeg{}}\HOLFreeVar{q}
\end{SaveVerbatim}
\newcommand{\HOLexerciseOneZeroTheoremsproblemTwoXXthm}{\UseVerbatim{HOLexerciseOneZeroTheoremsproblemTwoXXthm}}
\begin{SaveVerbatim}{HOLexerciseOneZeroTheoremsproblemThreeXXthm}
\HOLTokenTurnstile{} \HOLFreeVar{r} \HOLSymConst{\HOLTokenDisj{}} \HOLFreeVar{s}
\end{SaveVerbatim}
\newcommand{\HOLexerciseOneZeroTheoremsproblemThreeXXthm}{\UseVerbatim{HOLexerciseOneZeroTheoremsproblemThreeXXthm}}
\newcommand{\HOLexerciseOneZeroTheorems}{
\HOLThmTag{exercise10}{problem1_thm}\HOLexerciseOneZeroTheoremsproblemOneXXthm
\HOLThmTag{exercise10}{problem2_thm}\HOLexerciseOneZeroTheoremsproblemTwoXXthm
\HOLThmTag{exercise10}{problem3_thm}\HOLexerciseOneZeroTheoremsproblemThreeXXthm
}


\newpage

\HOLpagestyle

% ::::::::::::::::::::::::::::::::::::::::::::::::::::::::::::::::::::::::::
\subsection{Exercise 9 Theory}
\index{exercise9 Theory@\textbf  {exercise9 Theory}}
\begin{flushleft}
\textbf{Built:} \HOLexerciseNineDate \\[2pt]
\textbf{Parent Theories:} indexedLists, patternMatches
\end{flushleft}
% ::::::::::::::::::::::::::::::::::::::::::::::::::::::::::::::::::::::::::

% No datatypes

% No definitions

\subsection{Theorems}
\index{exercise9 Theory@\textbf  {exercise9 Theory}!Theorems}
% .....................................

\HOLexerciseNineTheorems



\cleardoublepage


\section{Chapter 10 Exercises}
\label{sec:Chap10}

\subsection{Problem Statement}
We use PAT_ASSUM to deal with assumptions in the goal orientated proofs of chapter 10. All of the theorems were proved without using PROVE_TAC.

\subsection{Relevant Code}

Exercise 10.4.1
\begin{lstlisting}[basicstyle=\tiny,]
val problem1_thm = 
TAC_PROOF (
([``!x:'a.P(x) ==> M(x)``,``(P:'a->bool)(s:'a)``], ``(M:'a->bool)(s:'a)``),
PAT_ASSUM ``!x.t`` (fn th => (ASSUME_TAC (SPEC ``s:'a`` th))) THEN
RES_TAC );
\end{lstlisting}

Exercise 10.4.2
\begin{lstlisting}[basicstyle=\tiny,]
val problem2_thm = 
TAC_PROOF (
([``p /\ q ==> r``,``r ==> s``,`` ~s``],``p ==> ~q``),
PAT_ASSUM ``r ==> s`` (fn th => ASSUME_TAC (IMP_ELIM th)) THEN
PAT_ASSUM ``~r:bool \/ s:bool`` (fn th => ASSUME_TAC (ONCE_REWRITE_RULE[DISJ_SYM] th)) THEN
PAT_ASSUM ``s:bool \/ ~r:bool`` (fn th => ASSUME_TAC (DISJ_IMP th)) THEN
RES_TAC THEN
PAT_ASSUM ``p:bool /\ q:bool ==> r:bool`` (fn th => ASSUME_TAC (ONCE_REWRITE_RULE[DISJ_SYM](IMP_ELIM th))) THEN
PAT_ASSUM ``r:bool \/ ~(p /\ q)`` (fn th => ASSUME_TAC (DISJ_IMP th)) THEN
RES_TAC THEN
(let
  val demorgan = SPEC ``q:bool`` (SPEC ``p:bool`` DE_MORGAN_THM)
in
  PAT_ASSUM ``~(p /\ q)``(fn th => ASSUME_TAC (REWRITE_RULE [] (DISJ_IMP (EQ_MP (CONJUNCT1 demorgan) (ASSUME ``~(p /\ q)``)))))
end) THEN
ASM_REWRITE_TAC []
);
\end{lstlisting}

Exercise 10.4.3
\begin{lstlisting}[basicstyle=\tiny,]
val problem3_thm = 
TAC_PROOF (
([`` ~(p /\ q)``, `` ~p ==> r``,`` ~q ==> s``],``r \/ s ``),
(let
  val demorgan = SPEC ``q:bool`` (SPEC ``p:bool`` DE_MORGAN_THM)
in
  PAT_ASSUM ``~(p /\ q)`` (fn th => ASSUME_TAC ((EQ_MP (CONJUNCT1 demorgan) (ASSUME ``~(p /\ q)``))))
end) THEN
PAT_ASSUM ``~p \/ ~q`` (fn th => ASSUME_TAC (REWRITE_RULE [] (DISJ_IMP th))) THEN
PAT_ASSUM ``p ==> ~q`` (fn th => ASSUME_TAC (IMP_TRANS th (ASSUME ``~q ==> s``))) THEN
PAT_ASSUM ``p ==> s`` (fn th => ASSUME_TAC (DISJ_IMP(ONCE_REWRITE_RULE[DISJ_SYM](IMP_ELIM th)))) THEN
PAT_ASSUM ``~s ==> ~p`` (fn th => ASSUME_TAC (IMP_TRANS th (ASSUME ``~p ==> r``))) THEN
PAT_ASSUM ``~s ==> r`` (fn th => ASSUME_TAC (REWRITE_RULE [](IMP_ELIM th))) THEN
ASM_REWRITE_TAC [DISJ_SYM]
)
\end{lstlisting}

\newpage 

\subsection{Test Results}

Exercise 10.4.1:
\verbatiminput{HOL/ex1041.trans}


Exercise 10.4.2:
\verbatiminput{HOL/ex1042.trans}


Exercise 10.4.3:
\verbatiminput{HOL/ex1043.trans}

\newpage

\HOLpagestyle


% ::::::::::::::::::::::::::::::::::::::::::::::::::::::::::::::::::::::::::
\subsection{Exercise 10 Theory}
\index{exercise10 Theory@\textbf  {exercise10 Theory}}
\begin{flushleft}
\textbf{Built:} \HOLexerciseOneZeroDate \\[2pt]
\textbf{Parent Theories:} indexedLists, patternMatches
\end{flushleft}
% ::::::::::::::::::::::::::::::::::::::::::::::::::::::::::::::::::::::::::

% No datatypes

% No definitions

\subsection{Theorems}
\index{exercise10 Theory@\textbf  {exercise10 Theory}!Theorems}
% .....................................

\HOLexerciseOneZeroTheorems

\HOLindex

\newpage 

\appendix

\section{Source Code for Exercise9 Theory}
\label{sec:source-code-exerc}

\verbatiminput{HOL/exercise9Script.sml}

\newpage

\section{Source Code for Exercise10 Theory}
\label{sec:source-code-exerc-1}

\verbatiminput{HOL/exercise10Script.sml}


\end{document}
