
\documentclass[11pt, twoside]{article}

%% ---------------------------------------------------
%% 634format specifies the format of our reports
%% ---------------------------------------------------
\usepackage{634format}

%% ---------------------------------------------------
%% enumerate 
%% ---------------------------------------------------
\usepackage{enumerate}

%% ---------------------------------------------------
%% listings is used for including our source code in reports
%% ---------------------------------------------------
\usepackage{listings}
\usepackage{textcomp}
\usepackage{indentfirst}

%% ---------------------------------------------------
%% Packages for math environments
%% ---------------------------------------------------
\usepackage{amsmath}

%% ---------------------------------------------------
%% Packages for URLs and hotlinks in the table of contents
%% and symbolic cross references using \ref
%% ---------------------------------------------------
\usepackage{hyperref}

%% ---------------------------------------------------
%% Packages for using HOL-generated macros and displays
%% ---------------------------------------------------
\usepackage{holtex}
\usepackage{holtexbasic}
\input{commands}

%% ---------------------------------------------------
%% Package for a table that will break across pages
%% ---------------------------------------------------
\usepackage{longtable}

%% ---------------------------------------------------
%% Packages to bring the contents of other files in
%% ---------------------------------------------------
\usepackage{import}
\usepackage{moreverb}

%% ---------------------------------------------------
%% parameter "language" is set to "ML"
%% ---------------------------------------------------
\lstset{language=ML}

\makeindex


\title{Project 4}
\author{\textbf{Alfred Murabito}}
\date{\today}

\begin{document}
\maketitle{}

\newpage

\begin{abstract}
   \noindent{}This report details results for the following exercises from \textit{Certified Security by Design Using Higher Order Logic}: 9.5.1, 9.5.2, 9.5.3, 10.4.1, 10.4.2, and 
10.4.3.  Chapter 9 of the textbook focuses on goal-oriented proofs while  Chapter 10 focuses on dealing with assumptions in goal-oriented proofs using PAT_ASSUM to match our supplied
terms to those in the assumptions..  Each of the exercises includes a problem statement, relevant code, execution transcripts, and our pretty-printed theorems. 
\end{abstract}

\newpage

\textbf{Acknowledgments}: I received no assistance with this project.

\newpage

\tableofcontents

\newpage

\section{Executive Summary}
\label{sec:executive-summary}

All requirements for this project have been satisfied.  A description of each exercise from Chapter 9 and 10 are included in the following chapters.  
Our HOL script files are contained in the HOL directory and pretty-print our theories to reports using EmiTex in the HOLReports directory. In the appendix
we also include all HOL source files.

\newpage


\section{Chapter 9 Exercises}
\label{sec:Chap9}

\subsection{Problem Statement}
Exercises 9.5.1 and 9.5.2 involve proving the absorption and constructiveDilemma theorems respectively. In exercise 9.5.3, we prove both theorems again using PROVE_TAC.

\subsection{Relevant Code}

Exercise 9.5.1
\begin{lstlisting}[basicstyle=\small,]
  val absorptionRule = 
TAC_PROOF (
( [], ``!(p:bool)(q:bool).(p ==> q) ==> p ==> (p /\ q)``),
REPEAT STRIP_TAC THENL
[(ACCEPT_TAC (ASSUME ``p:bool``)),
(ACCEPT_TAC (MP (ASSUME ``p:bool==>q:bool``) (ASSUME ``p:bool``)))])
\end{lstlisting}

Exercise 9.5.2
\begin{lstlisting}[basicstyle=\small,]
val constructiveDilemmaRule = 
TAC_PROOF (
( [], ``!(p:bool)(q:bool)(r:bool)(s:bool).(p==>q)/\(r==>s)==>(p\/r)==>(q\/s)``),
(REPEAT STRIP_TAC THENL
[(RES_TAC THEN (ACCEPT_TAC (DISJ1 (ASSUME ``q:bool``) ``s:bool``))),
(RES_TAC THEN (ACCEPT_TAC (DISJ2 ``q:bool`` (ASSUME ``s:bool``))))])
);
\end{lstlisting}

Exercise 9.5.3
\begin{lstlisting}[basicstyle=\small,]
val absorptionRule2 = 
TAC_PROOF (
( [], ``!(p:bool)(q:bool).(p ==> q) ==> p ==> (p /\ q)``),
PROVE_TAC [])

val constructiveDilemmaRule2 = 
TAC_PROOF (
( [], ``!(p:bool)(q:bool)(r:bool)(s:bool).(p==>q) /\ (r==>s) ==> p ==> (p /\ q)``),
PROVE_TAC [])
\end{lstlisting}

\newpage 

\subsection{Test Results}


Exercise 9.5.1:
\verbatiminput{HOL/ex951.trans}


Exercise 9.5.2:
\verbatiminput{HOL/ex952.trans}


Exercise 9.5.3:
\verbatiminput{HOL/ex953.trans}



\input{HOLReports/HOLexerciseNine}
\input{HOLReports/HOLexerciseOneZero}

\newpage

\HOLpagestyle

% ::::::::::::::::::::::::::::::::::::::::::::::::::::::::::::::::::::::::::
\subsection{Exercise 9 Theory}
\index{exercise9 Theory@\textbf  {exercise9 Theory}}
\begin{flushleft}
\textbf{Built:} \HOLexerciseNineDate \\[2pt]
\textbf{Parent Theories:} indexedLists, patternMatches
\end{flushleft}
% ::::::::::::::::::::::::::::::::::::::::::::::::::::::::::::::::::::::::::

% No datatypes

% No definitions

\subsection{Theorems}
\index{exercise9 Theory@\textbf  {exercise9 Theory}!Theorems}
% .....................................

\HOLexerciseNineTheorems



\cleardoublepage


\section{Chapter 10 Exercises}
\label{sec:Chap10}

\subsection{Problem Statement}
We use PAT_ASSUM to deal with assumptions in the goal orientated proofs of chapter 10. All of the theorems were proved without using PROVE_TAC.

\subsection{Relevant Code}

Exercise 10.4.1
\begin{lstlisting}[basicstyle=\tiny,]
val problem1_thm = 
TAC_PROOF (
([``!x:'a.P(x) ==> M(x)``,``(P:'a->bool)(s:'a)``], ``(M:'a->bool)(s:'a)``),
PAT_ASSUM ``!x.t`` (fn th => (ASSUME_TAC (SPEC ``s:'a`` th))) THEN
RES_TAC );
\end{lstlisting}

Exercise 10.4.2
\begin{lstlisting}[basicstyle=\tiny,]
val problem2_thm = 
TAC_PROOF (
([``p /\ q ==> r``,``r ==> s``,`` ~s``],``p ==> ~q``),
PAT_ASSUM ``r ==> s`` (fn th => ASSUME_TAC (IMP_ELIM th)) THEN
PAT_ASSUM ``~r:bool \/ s:bool`` (fn th => ASSUME_TAC (ONCE_REWRITE_RULE[DISJ_SYM] th)) THEN
PAT_ASSUM ``s:bool \/ ~r:bool`` (fn th => ASSUME_TAC (DISJ_IMP th)) THEN
RES_TAC THEN
PAT_ASSUM ``p:bool /\ q:bool ==> r:bool`` (fn th => ASSUME_TAC (ONCE_REWRITE_RULE[DISJ_SYM](IMP_ELIM th))) THEN
PAT_ASSUM ``r:bool \/ ~(p /\ q)`` (fn th => ASSUME_TAC (DISJ_IMP th)) THEN
RES_TAC THEN
(let
  val demorgan = SPEC ``q:bool`` (SPEC ``p:bool`` DE_MORGAN_THM)
in
  PAT_ASSUM ``~(p /\ q)``(fn th => ASSUME_TAC (REWRITE_RULE [] (DISJ_IMP (EQ_MP (CONJUNCT1 demorgan) (ASSUME ``~(p /\ q)``)))))
end) THEN
ASM_REWRITE_TAC []
);
\end{lstlisting}

Exercise 10.4.3
\begin{lstlisting}[basicstyle=\tiny,]
val problem3_thm = 
TAC_PROOF (
([`` ~(p /\ q)``, `` ~p ==> r``,`` ~q ==> s``],``r \/ s ``),
(let
  val demorgan = SPEC ``q:bool`` (SPEC ``p:bool`` DE_MORGAN_THM)
in
  PAT_ASSUM ``~(p /\ q)`` (fn th => ASSUME_TAC ((EQ_MP (CONJUNCT1 demorgan) (ASSUME ``~(p /\ q)``))))
end) THEN
PAT_ASSUM ``~p \/ ~q`` (fn th => ASSUME_TAC (REWRITE_RULE [] (DISJ_IMP th))) THEN
PAT_ASSUM ``p ==> ~q`` (fn th => ASSUME_TAC (IMP_TRANS th (ASSUME ``~q ==> s``))) THEN
PAT_ASSUM ``p ==> s`` (fn th => ASSUME_TAC (DISJ_IMP(ONCE_REWRITE_RULE[DISJ_SYM](IMP_ELIM th)))) THEN
PAT_ASSUM ``~s ==> ~p`` (fn th => ASSUME_TAC (IMP_TRANS th (ASSUME ``~p ==> r``))) THEN
PAT_ASSUM ``~s ==> r`` (fn th => ASSUME_TAC (REWRITE_RULE [](IMP_ELIM th))) THEN
ASM_REWRITE_TAC [DISJ_SYM]
)
\end{lstlisting}

\newpage 

\subsection{Test Results}

Exercise 10.4.1:
\verbatiminput{HOL/ex1041.trans}


Exercise 10.4.2:
\verbatiminput{HOL/ex1042.trans}


Exercise 10.4.3:
\verbatiminput{HOL/ex1043.trans}

\newpage

\HOLpagestyle


% ::::::::::::::::::::::::::::::::::::::::::::::::::::::::::::::::::::::::::
\subsection{Exercise 10 Theory}
\index{exercise10 Theory@\textbf  {exercise10 Theory}}
\begin{flushleft}
\textbf{Built:} \HOLexerciseOneZeroDate \\[2pt]
\textbf{Parent Theories:} indexedLists, patternMatches
\end{flushleft}
% ::::::::::::::::::::::::::::::::::::::::::::::::::::::::::::::::::::::::::

% No datatypes

% No definitions

\subsection{Theorems}
\index{exercise10 Theory@\textbf  {exercise10 Theory}!Theorems}
% .....................................

\HOLexerciseOneZeroTheorems

\HOLindex

\newpage 

\appendix

\section{Source Code for Exercise9 Theory}
\label{sec:source-code-exerc}

\verbatiminput{HOL/exercise9Script.sml}

\newpage

\section{Source Code for Exercise10 Theory}
\label{sec:source-code-exerc-1}

\verbatiminput{HOL/exercise10Script.sml}


\end{document}
