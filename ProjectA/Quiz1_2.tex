%% ---------------------------------------------------
%% Notice that we use the "report" class instead of "article"
%% ---------------------------------------------------
\documentclass{report}

\title{Assurance Foundations Quiz 1}
\author{Alfred Murabito}
\date{\today}

%% ---------------------------------------------------
%% 634format specifies the format of our reports
%% ---------------------------------------------------
\usepackage{634format}

%% ---------------------------------------------------
%% enumerate 
%% ---------------------------------------------------
\usepackage{enumerate}

%% ---------------------------------------------------
%% listings is used for including our source code in reports
%% textcomp provides additional symbols
%% ---------------------------------------------------
\usepackage{listings}
\usepackage{textcomp}
\usepackage{import}

%% ---------------------------------------------------
%% Packages for math environments
%% -------------------------------\--------------------
\usepackage{amsmath}

%% ---------------------------------------------------
%% Packages for URLs and hotlinks in the table of contents
%% and symbolic cross references using \ref
%% ---------------------------------------------------
\usepackage{hyperref}

%% ---------------------------------------------------
%% Packages for using HOL-generated macros and displays
%% ---------------------------------------------------
\usepackage{holtex}
\usepackage{holtexbasic}
% =====================================================================
%
% Macros for typesetting the HOL system manual
%
% =====================================================================

% ---------------------------------------------------------------------
% Abbreviations for words and phrases
% ---------------------------------------------------------------------

\newcommand\TUTORIAL{{\footnotesize\sl TUTORIAL}}
\newcommand\DESCRIPTION{{\footnotesize\sl DESCRIPTION}}
\newcommand\REFERENCE{{\footnotesize\sl REFERENCE}}
\newcommand\LOGIC{{\footnotesize\sl LOGIC}}
\newcommand\LIBRARIES{{\footnotesize\sl LIBRARIES}}
\usepackage{textcomp}

\newcommand{\bs}{\texttt{\char'134}} % backslash
\newcommand{\lb}{\texttt{\char'173}} % left brace
\newcommand{\rb}{\texttt{\char'175}} % right brace
\newcommand{\td}{\texttt{\char'176}} % tilde
\newcommand{\lt}{\texttt{\char'74}} % less than
\newcommand{\gt}{\texttt{\char'76}} % greater than
\newcommand{\dol}{\texttt{\char'44}} % dollar
\newcommand{\pipe}{\texttt{\char'174}}
\newcommand{\apost}{\texttt{\textquotesingle}}
% double back quotes ``
\newcommand{\dq}{\texttt{\char'140\char'140}}
%These macros were included by slind:

\newcommand{\holquote}[1]{\dq#1\dq}

\def\HOL{\textsc{Hol}}
\def\holn{\HOL}  % i.e. hol n(inety-eight), no digits in
                 % macro names is a bit of a pain; deciding to do away
                 % with hol98 nomenclature means that we just want to
                 % write HOL for hol98.
\def\holnversion{Kananaskis-11}
\def\holnsversion{Kananaskis~11} % version with space rather than hyphen
\def\LCF{{\small LCF}}
\def\LCFLSM{{\small LCF{\kern-.2em}{\normalsize\_}{\kern0.1em}LSM}}
\def\PPL{{\small PP}{\kern-.095em}$\lambda$}
\def\PPLAMBDA{{\small PPLAMBDA}}
\def\ML{{\small ML}}
\def\holmake{\texttt{Holmake}}

\newcommand\ie{\mbox{\textit{i{.}e{.}}}}
\newcommand\eg{\mbox{\textit{e{.}g{.}}}}
\newcommand\viz{\mbox{viz{.}}}
\newcommand\adhoc{\mbox{\it ad hoc}}
\newcommand\etal{{\it et al.\/}}
% NOTE: \etc produces wrong spacing if used between sentences, that is
% like here \etc End such sentences with non-macro etc.
\newcommand\etc{\mbox{\textit{etc{.}}}}

% ---------------------------------------------------------------------
% Simple abbreviations and macros for mathematical typesetting
% ---------------------------------------------------------------------

\newcommand\fun{{\to}}
\newcommand\prd{{\times}}

\newcommand\conj{\ \wedge\ }
\newcommand\disj{\ \vee\ }
\newcommand\imp{ \Rightarrow }
\newcommand\eqv{\ \equiv\ }
\newcommand\cond{\rightarrow}
\newcommand\vbar{\mid}
\newcommand\turn{\ \vdash\ } % FIXME: "\ " results in extra space
\newcommand\hilbert{\varepsilon}
\newcommand\eqdef{\ \equiv\ }

\newcommand\natnums{\mbox{${\sf N}\!\!\!\!{\sf N}$}}
\newcommand\bools{\mbox{${\sf T}\!\!\!\!{\sf T}$}}

\newcommand\p{$\prime$}
\newcommand\f{$\forall$\ }
\newcommand\e{$\exists$\ }

\newcommand\orr{$\vee$\ }
\newcommand\negg{$\neg$\ }

\newcommand\arrr{$\rightarrow$}
\newcommand\hex{$\sharp $}

\newcommand{\uquant}[1]{\forall #1.\ }
\newcommand{\equant}[1]{\exists #1.\ }
\newcommand{\hquant}[1]{\hilbert #1.\ }
\newcommand{\iquant}[1]{\exists ! #1.\ }
\newcommand{\lquant}[1]{\lambda #1.\ }

\newcommand{\leave}[1]{\\[#1]\noindent}
\newcommand\entails{\mbox{\rule{.3mm}{4mm}\rule[2mm]{.2in}{.3mm}}}

% ---------------------------------------------------------------------
% Font-changing commands
% ---------------------------------------------------------------------

\newcommand{\theory}[1]{\hbox{{\small\tt #1}}}
\newcommand{\theoryimp}[1]{\texttt{#1}}

\newcommand{\con}[1]{{\sf #1}}
\newcommand{\rul}[1]{{\tt #1}}
\newcommand{\ty}[1]{\textsl{#1}}

\newcommand{\ml}[1]{\mbox{{\def\_{\char'137}\texttt{#1}}}}
\newcommand{\holtxt}[1]{\ml{#1}}
\newcommand\ms{\tt}
\newcommand{\s}[1]{{\small #1}}

\newcommand{\pin}[1]{{\bf #1}}
% FIXME: for multichar symbols \mathit should be used.
\def\m#1{\mbox{\normalsize$#1$}}

% ---------------------------------------------------------------------
% Abbreviations for particular mathematical constants etc.
% ---------------------------------------------------------------------

\newcommand\T{\con{T}}
\newcommand\F{\con{F}}
\newcommand\OneOne{\con{One\_One}}
\newcommand\OntoSubset{\con{Onto\_Subset}}
\newcommand\Onto{\con{Onto}}
\newcommand\TyDef{\con{Type\_Definition}}
\newcommand\Inv{\con{Inv}}
\newcommand\com{\con{o}}
\newcommand\Id{\con{I}}
\newcommand\MkPair{\con{Mk\_Pair}}
\newcommand\IsPair{\con{Is\_Pair}}
\newcommand\Fst{\con{Fst}}
\newcommand\Snd{\con{Snd}}
\newcommand\Suc{\con{Suc}}
\newcommand\Nil{\con{Nil}}
\newcommand\Cons{\con{Cons}}
\newcommand\Hd{\con{Hd}}
\newcommand\Tl{\con{Tl}}
\newcommand\Null{\con{Null}}
\newcommand\ListPrimRec{\con{List\_Prim\_Rec}}


\newcommand\SimpRec{\con{Simp\_Rec}}
\newcommand\SimpRecRel{\con{Simp\_Rec\_Rel}}
\newcommand\SimpRecFun{\con{Simp\_Rec\_Fun}}
\newcommand\PrimRec{\con{Prim\_Rec}}
\newcommand\PrimRecRel{\con{Prim\_Rec\_Rel}}
\newcommand\PrimRecFun{\con{Prim\_Rec\_Fun}}

\newcommand\bool{\ty{bool}}
\newcommand\num{\ty{num}}
\newcommand\ind{\ty{ind}}
\newcommand\lst{\ty{list}}

% ---------------------------------------------------------------------
% \minipagewidth = \textwidth minus 1.02 em
% ---------------------------------------------------------------------

\newlength{\minipagewidth}
\setlength{\minipagewidth}{\textwidth}
\addtolength{\minipagewidth}{-1.02em}

% ---------------------------------------------------------------------
% Environment for the items on the title page of a case study
% ---------------------------------------------------------------------

\newenvironment{inset}[1]{\noindent{\large\bf #1}\begin{list}%
{}{\setlength{\leftmargin}{\parindent}%
\setlength{\topsep}{-.1in}}\item }{\end{list}\vskip .4in}

% ---------------------------------------------------------------------
% Macros for little HOL sessions displayed in boxes.
%
% Usage: (1) \setcounter{sessioncount}{1} resets the session counter
%
%        (2) \begin{session}\begin{verbatim}
%             .
%              < lines from hol session >
%             .
%            \end{verbatim}\end{session}
%
%            typesets the session in a numbered box.
% ---------------------------------------------------------------------

\newlength{\hsbw}
\setlength{\hsbw}{\textwidth}
\addtolength{\hsbw}{-\arrayrulewidth}
\addtolength{\hsbw}{-\tabcolsep}
\newcommand\HOLSpacing{13pt}

\newcounter{sessioncount}
\setcounter{sessioncount}{0}

\newenvironment{session}{\begin{flushleft}
 \refstepcounter{sessioncount}
 \begin{tabular}{@{}|c@{}|@{}}\hline
 \begin{minipage}[b]{\hsbw}
 \vspace*{-.5pt}
 \begin{flushright}
 \rule{0.01in}{.15in}\rule{0.3in}{0.01in}\hspace{-0.35in}
 \raisebox{0.04in}{\makebox[0.3in][c]{\footnotesize\sl \thesessioncount}}
 \end{flushright}
 \vspace*{-.55in}
 \begingroup\small\baselineskip\HOLSpacing}{\endgroup\end{minipage}\\ \hline
 \end{tabular}
 \end{flushleft}}

% ---------------------------------------------------------------------
% Macro for boxed ML functions, etc.
%
% Usage: (1) \begin{holboxed}\begin{verbatim}
%               .
%               < lines giving names and types of mk functions >
%               .
%            \end{verbatim}\end{holboxed}
%
%            typesets the given lines in a box.
%
%            Conventions: lines are left-aligned under the "g" of begin,
%            and used to highlight primary reference for the ml function(s)
%            that appear in the box.
% ---------------------------------------------------------------------

\newenvironment{holboxed}{\begin{flushleft}
  \begin{tabular}{@{}|c@{}|@{}}\hline
  \begin{minipage}[b]{\hsbw}
% \vspace*{-.55in}
  \vspace*{.06in}
  \begingroup\small\baselineskip\HOLSpacing}{\endgroup\end{minipage}\\ \hline
  \end{tabular}
  \end{flushleft}}

% ---------------------------------------------------------------------
% Macro for unboxed ML functions, etc.
%
% Usage: (1) \begin{hol}\begin{verbatim}
%               .
%               < lines giving names and types of mk functions >
%               .
%            \end{verbatim}\end{hol}
%
%            typesets the given lines exactly like {boxed}, except there's
%            no box.
%
%            Conventions: lines are left-aligned under the "g" of begin,
%            and used to display ML code in verbatim, left aligned.
% ---------------------------------------------------------------------

\newenvironment{hol}{\begin{flushleft}
 \begin{tabular}{c@{}@{}}
 \begin{minipage}[b]{\hsbw}
% \vspace*{-.55in}
 \vspace*{.06in}
 \begingroup\small\baselineskip\HOLSpacing}{\endgroup\end{minipage}\\
 \end{tabular}
 \end{flushleft}}

% ---------------------------------------------------------------------
% Emphatic brackets
% ---------------------------------------------------------------------

\newcommand\leb{\lbrack\!\lbrack}
\newcommand\reb{\rbrack\!\rbrack}


% ---------------------------------------------------------------------
% Quotations
% ---------------------------------------------------------------------


%These macros were included by ap; they are used in Chapters 9 and 10
%of the HOL DESCRIPTION

\newcommand{\inds}%standard infinite set
 {\mbox{\rm I}}

\newcommand{\ch}%standard choice function
 {\mbox{\rm ch}}

\newcommand{\den}[1]%denotational brackets
 {[\![#1]\!]}

\newcommand{\two}%standard 2-element set
 {\mbox{\rm 2}}


\begin{document}

%% --------------------------------------------------- the listings
%% parameter "language" is set to "ML"
%% ---------------------------------------------------
\lstset{language=ML}


\maketitle{}


\tableofcontents{}

\begin{acknowledgments}
 I received no external help for this report.
\end{acknowledgments}

\chapter{Executive Summary}
\label{cha:executive-summary}

\textbf{All requirements for this project are satisfied}.
Specifically,
\begin{description}
\item[Report Contents] \ \\
  Our report has the following content:
  \begin{enumerate}[{}]
  \item Chapter~\ref{cha:executive-summary}: Executive Summary
  \item Chapter~\ref{cha:report-body}: Report Body
    \begin{enumerate}[{}]
    \item Section~\ref{sec:problem-statement}: Problem statement
    \end{enumerate}
  \item Chapter~\ref{cha:special-coding-tests} Code and Tests
    \begin{enumerate}[{}]
    \item Section~\ref{sec:relevant-code}: Relevant code
    \item Section~\ref{sec:tests}: Test results
    \end{enumerate}
  \item Chapter~\ref{cha:source-code-sample}: Source Code
  \item Chapter~\ref{cha:addendum}: Addendum
  \end{enumerate}
\end{description}


\chapter{Report Body}
\label{cha:report-body}


\section{Problem Statement}
\label{sec:problem-statement}

The Fleece'm'n-How law firm Generating Law has a need to keep a ratio of 
male to female workers above 50 percent as well as a combined total experience of at least 60,000 hours. 
Mr How has decided to analyze the total experience and gender ratio of the Junior Partners since
he heard it is best if the firm is comprised of young workers. 

\begin{enumerate}
\item To represent the table of data in ML, I used a (string * string * string * int) list data structure.

\item To calculate the total hours of all junior partners, I created a function
that accepts the job status of interest and a list of employee entries with
the important fields being job status, gender, and experience. It recursively iterates through
each element in the list using constructors and pattern matching and adds an elements experience
to the running total if the job status matches (job status, experience, and gender are extracted
from the list entry using pattern matching).

\item To get the male/female split of a given job status,I used a similar approach as before to 
iterate through each entry in the list. If the entry meets the criteria for a male or female
of the given job status, then the corresponding element in the tuple(male or female) will
be incremented in the tuple returned using pattern matching.

\end{enumerate}


\chapter{Special Coding and Tests}
\label{cha:special-coding-tests}


\section{Relevant Code}
\label{sec:relevant-code}

\subsection{Employee Firm Data}
\label{sec:employee-firm-data}

First piece of code we need is a definition in ML of the table containing
all of the relevant employee information.

The following code is from \emph{table.sml}
\lstinputlisting{ML/table.sml}

\subsection{Calculate Total Hours Experience}
\label{sec:calc-total-hours}

\lstset{frameround=fftt}
\begin{lstlisting}[frame=tRBL]

(* Calculate the total hours in experience of  *) 
(* a given seniority (status)                  *)

fun calc_h (status:string) [] = 0
  | calc_h (status) (head::table:(string*string*string*int)list) =
  let
    val (name, gender, j_status, exp) = head
    val match:bool = (status = j_status)
  in
    if match then (2080*exp)+(calc_h status table) else (calc_h status table)
  end;

(* For calculating total hours for all Jr. Partners, use below *)
calc_h ``Jr. Partner'' table;

\end{lstlisting}
\pagebreak
\subsection{Get Male/Female Split}
\label{sec:get-malefemale-split}

\lstset{frameround=fftt}
\begin{lstlisting}[frame=tRBL]

(* Calculate the male/female split *)

fun get_male_female(status:string) [] = (0, 0)
  | get_male_female(status:string) (head::table) =
  let
    (* Use pattern matching to get elmenets of list entry *)
    val (name, gender, j_status, exp) = head
    val f_match:bool = ((status = j_status) andalso (gender = "F"))
    val m_match:bool = ((status = j_status) andalso (gender = "M"))
    val (p_male, p_fem) = get_male_female status table
  in
    ((if m_match then 1 else 0)+p_male,(if f_match then 1 else 0)+p_fem)
  end;

(* To get the male female split for Jr. Partners, use below *)
get_male_female ``Jr. Partner'' table;

\end{lstlisting}

\section{Test Results}
\label{sec:tests}


\subsection{Test results for employee data}
\label{sec:test-results-empl-1}
%%-------------------------------------------------------------
%% Notice how we set sessioncount to 0 so that the first HOL
%% session is numbered starting from 1
%%-------------------------------------------------------------

\setcounter{sessioncount}{0}

 \import{ML/}{calch.trans}


\subsection{Test results for calch}
\label{sec:test-results-calch}

\begin{session}
  \begin{scriptsize}
\begin{verbatim}

> fun calc_h (status:string) [] = 0
  | calc_h (status) (head::table:(string*string*string*int)list) =
  let
    val (name, gender, j_status, exp) = head
    val match:bool = (status = j_status)
  in
    if match then (2080*exp)+(calc_h status table) else (calc_h status table)
  end;


calc_h "Jr. Partner" table;
# # # # # # # val calc_h = fn: string -> (string * string * string * int) list -> int
> > # # val it = 434720: int


\end{verbatim}
  \end{scriptsize}
\end{session}

There were a calculated number of 434,720 total accumulated hours of experience for all Jr. Partners.
This is substantially more than the recommended amount.

\subsection{Test results for male/female split}
\label{sec:test-results-malef}


\begin{session}
  \begin{scriptsize}
\begin{verbatim}

> (* Calculate the male/female split *)

fun get_male_female(status:string) [] = (0, 0)
  | get_male_female(status:string) (head::table) =
  let
    (* Use pattern matching to get elmenets of list entry *)
    val (name, gender, j_status, exp) = head
    val f_match:bool = ((status = j_status) andalso (gender = "F"))
    val m_match:bool = ((status = j_status) andalso (gender = "M"))
    val (p_male, p_fem) = get_male_female status table
  in
    ((if m_match then 1 else 0)+p_male,(if f_match then 1 else 0)+p_fem)
  end;

get_male_female "Jr. Partner" table;
# # # # # # # # # # # # val get_male_female = fn:
   string -> ('a * string * string * 'b) list -> int * int
> > # # val it = (11, 3): int * int


\end{verbatim}
  \end{scriptsize}
\end{session}

The split between the Jr. Partners is 11 males to 3 females, about 79 percent 
male. This is more than adequate for the gender ratio requirement. 
Hiring more Jr. Partners seems like a good long term plan due to the high
experience and male gender ratio.

%% ------------------------------------------
%% this restarts the section numbering and
%% changes chapter numbering to letters starting
%% with A
%% ------------------------------------------
\appendix{} 


\chapter{Source Code for Sample Exercise}
\label{cha:source-code-sample}

The following code is from \emph{quiz1.sml}
\lstinputlisting{ML/quiz1.sml}

\chapter{Addendum}
\label{cha:addendum}

\section{Relevant Code}
\label{sec:relevant-code-1}


The following code is from \emph{modquiz1.sml}
\lstinputlisting{ML/modquiz1.sml}

\section{Test Cases}
\label{sec:test-cases}

\subsection{Test cases for modified calch}
\label{sec:test-cases-modified}


\begin{session}
  \begin{scriptsize}
\begin{verbatim}

> fun calc_h (status:string) [] = 0
  | calc_h (status) (head::table:(string*string*string*int)list) =
  let
    val (name, gender, j_status, exp) = head
    val match:bool = (status = j_status)
  in
    if match then (2080*exp)+(calc_h status table) else (calc_h status table)
  end;

(* Modifed line below for Associate *)
(* calc_h "Jr. Partner" table; *)
calc_h "Associate" table;

val calc_h = fn: string -> (string * string * string * int) list -> int
> > val it = 326560: int > plus 0 0;

\end{verbatim}
  \end{scriptsize}
\end{session}

The Associates have a combined total of 326,560 hours of experience.

\subsection{Test cases for modified male/female split}
\label{sec:test-cases-modified-2}


\begin{session}
  \begin{scriptsize}
\begin{verbatim}

> (* Calculate the male/female split *)

fun get_male_female(status:string) [] = (0, 0)
  | get_male_female(status:string) (head::table) =
  let
    (* Use pattern matching to get elmenets of list entry *)
    val (name, gender, j_status, exp) = head
    val f_match:bool = ((status = j_status) andalso (gender = "F"))
    val m_match:bool = ((status = j_status) andalso (gender = "M"))
    val (p_male, p_fem) = get_male_female status table
  in
    ((if m_match then 1 else 0)+p_male,(if f_match then 1 else 0)+p_fem)
  end;

(* Modifed line below for Associate *)
(* get_male_female "Jr. Partner" table; *)
get_male_female "Associate" table;
# # # # # # # # # # # # val get_male_female = fn:string -> ('a * string * string * 'b) list -> int * int
> > val it = (14, 12): int * int


\end{verbatim}
  \end{scriptsize}
\end{session}

The Jr. Partners have a much higher ratio of males to females(11 to 3) than the Associates (14 to 12). In addition, the total hours of 
experience between the Jr. Partners is much higher than for the Associates. Hiring more Jr. Partners appears to be the sound option over
more Associates to prevent risk of the male gender ratio dropping and stay above the recommendation. 

\end{document}